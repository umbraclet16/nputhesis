%%UTF-8
\documentclass[twoside,UTF8]{nputhesis}
%\documentclass[oneside]{nputhesis}

\usepackage{amsmath}
\usepackage{amsfonts}
\usepackage{booktabs}
\usepackage{multirow}
\usepackage{graphicx}

\usepackage{lipsum}

\schoolno{10699}
%\classno{}
%\secretlevel{}
\title[Research on Vision-based Object Detection and Localization for Rotorcraft UAVs]{基于视觉的旋翼无人机目标检测定位方法研究}
\author[You Fang]{方酉}
\authorno{2015201548}
\major[Control Theory and Control Engineering]{控制理论与控制工程}
\supervisor[Weiguo Zhang]{章卫国}
\applydate[March 2018]{2018~年~3~月}
%\support{本文研究得到某某基金(编号:XXXXXXX)资助。}

\begin{document}
\makecover  % 中英文封面
\frontmatter

%==============================
% 摘要
% 中文摘要
\begin{abstract}  

    \begin{keywords}
        立体匹配,深度学习,卷积神经网络,旋翼无人机,目标定位 
    \end{keywords}
\end{abstract}

% 英文摘要
\begin{Abstract}

    \begin{Keywords}
        Stereo Matching, Deep Learning, Convolutional Neural Network, Rotorcraft UAV, Object localization
    \end{Keywords}
\end{Abstract}
%==============================
% 目录 (需要编译两次才能正确更新目录)
\tableofcontents 
%==============================
\mainmatter  % 
%==============================
%          乱七八糟的想法记在这
%>>>>>>>>>>>>>>>>>>>>>>>>>>>>>>
% 每章第一节加个引言?

% 4.2 介绍Tensorflow的部分可能要放在第三章。
%((TensorFlow是谷歌基于DistBelief开发的第二代人工智能学习系统,其采用数据流图(data flow graph)的形式来建立模型,图中每个节点表示一个数学运算或数据的输入/输出,每条边则表示节点间的输入输出关系,边上传输的是多维数组数据,即张量(Tensor)。TensorFlow具有自动求微分的功能,非常适合于深度学习等基于梯度的机器学习算法。))

% 计算机配置也放在第三章。
%本实验使用的计算机配置见表\ref{}。
%
%\begin{table}[htbp]
%	\centering
%	\caption{实验运行环境}
%	\label{tab:4_2_experiment_environment}
%	\begin{small} %{scriptsize}
%		\begin{tabular}{|l|c|}\hline
%			项目  & 内容  \\\hline
%			%---------------------------------------------------------
%			CPU & Intel(R) Xeon(R) CPU E5-2630 v2 @2.60GHz, 12核 \\
%			内存 & 24G \\
%			显卡 & NVidia GeForce GTX TITAN X \\
%			显存 & 12G \\
%			CUDA & V7.5 \\
%			操作系统 & Ubuntu 16.04.1 LTS \\\hline
%		\end{tabular}
%	\end{small} %{scriptsize}
%\end{table}


%<<<<<<<<<<<<<<<<<<<<<<<<<<<<<<
%==============================
% 综述
% 第一章 综述

\chapter{综述}

\section{研究背景及意义}

提一下目标定位一般有什么方法,然后引到双目立体视觉。
% 目标定位的研究概况?

%------------------------------------------------------------------------------------
\section{目标检测的研究概况}

\section{立体匹配的研究概况}
% 研究现状,研究目标,研究内容

% ref: GC-Net ch2.
使用立体图像对来计算深度信息的研究已经进行了很多年\cite{Barnard:1982:CS:356893.356896},而立体匹配是双目立体视觉中最重要的部分。Scharstein和Szeliski于2002年发表的一篇综述论文\cite{Scharstein2002}对立体匹配算法进行了分类,将立体匹配分为四个步骤:匹配代价计算(matching cost computation)、代价支持聚合(cost support aggregation)、视差计算和优化(disparity computation and optimization)和视差精炼(disparity refinement)。大部分立体匹配算法都可以归类为这四个步骤的子集。

Scharstein的综述还介绍了Middlebury的第一个数据集和相关的评价指标,并提供了使用结构光测量方法获得的数据集的测量真值(groundtruth)。KITTI数据集\cite{Geiger2012,Menze_2015_CVPR}是一个比Middlebury更大的数据集,其图像是在一辆行驶中的汽车上采集的,真实数据通过激光雷达获得。这些数据集推动了立体匹配算法的发展。
近年来随着深度学习的流行,出现了很多基于卷积神经网络的立体匹配方法。在此我们将不使用深度学习模型的方法归为传统方法,和基于深度学习的方法分别进行介绍。

\subsection{传统立体匹配方法}
% ref: SGM paper ch1.
传统立体匹配方法大致可分为局部方法和全局方法两类。局部方法至少包含匹配代价计算、代价聚合和视差计算三个部分。匹配代价是一种用来衡量潜在匹配区域内像素差异性的手段。常见的匹配代价包括绝对差值、平方差值、截断差值和对采样不敏感的差值\cite{Birchfield1999}等。这些代价对于光照度差异比较敏感,可以采用基于图像梯度的匹配代价进行改善\cite{SGM_3}。为了处理图像间复杂的光照度关系,互信息被引入了计算机视觉领域\cite{Viola1997},并应用于立体匹配中\cite{SGM_6}。匹配代价聚合将一定范围内的匹配代价联系起来,通常仅仅是在某个固定尺寸的窗口内对所有的匹配代价进行求和\cite{Hirschmüller2002}。也有一些方法根据颜色相似度和距离中心的距离对窗口内各个像素进行加权\cite{SGM_10}。另外一种方式是根据基于相同亮度或颜色进行分割得到的结果来选择代价聚合的区域\cite{Zitnick:2004}。视差计算是通过选择最低匹配代价的视差完成的\cite{Hirschmüller2002},即“赢者通吃”(Winner Takes All)。局部方法计算量小,算法效率高,但对于低纹理、遮挡和视差不连续的区域匹配效果较差。

全局匹配方法一般会跳过代价聚集的步骤,定义一个包含数据项和平滑项的能量函数,将匹配转化为最小化能量函数的优化问题。平滑项的加入有助于获得分段光滑的视差。有些算法还会额外加入惩罚遮挡\cite{SGM_13}、强制左右图一致性\cite{Zitnick:2004}的数据项。最小化能量函数的策略有很多,包括图割(Graph Cuts)\cite{SGM_13},置信传播(Belief Propagation)\cite{SGM_3},动态规划\cite{VanMeerbergen2002}等。\cite{MRF}和\cite{CRF}分别使用数据集的真实深度图来训练马尔科夫随机场(MRF)和条件随机场(CRF)模型。

全局匹配算法的优点是匹配准确度较高,但其计算量很大,相应的参数设置也比较复杂。半全局匹配(SGM)\cite{Hirschmuller08}是一种对全局算法的有效近似,通过多个方向上的动态规划降低了计算复杂度,既能获得媲美全局算法的的匹配效果,又能保证较高的效率。

\subsection{基于深度学习的立体匹配方法}
% ref: CRL paper, ch2; SsSMNet paper ch2.
传统立体匹配方法对于匹配无纹理区域、反射平面、重复图案、细微结构等表现不佳。很多算法通过池化、基于梯度的正则化\cite{geiger2010efficient, hirschmuller2005accurate}来解决这些问题,但通常需要在平滑表面和检测细节方面进行折中。而深度学习模型已经在物体分类\cite{krizhevsky2012imagenet}、检测\cite{girshick2014rich}和语义分割\cite{badrinarayanan2015segnet}等方面取得了巨大成功,显示出了深度卷积神经网络在理解语义方面的有效性。事实上,近年来深度卷积神经网络也已被成功应用于立体匹配,并大大推动了该领域的发展。基于CNN的匹配方法大致可以分为三类:匹配代价学习、
正则学习(regularity learning)和端到端(End-to-end)的视差学习。

匹配代价学习:使用卷积神经网络获得的图像特征取代人工设计的匹配代价指标。Han等人\cite{Han_2015_CVPR}提出了MatchNet,该模型从成对的图像块中提取特征,然后用一个决策模块来衡量相似度。Zbontar等人\cite{zbontar2016stereo}设计了MC-CNN模型,训练了一个用于匹配的二元分类器。Luo等人\cite{luo2016efficient}提出了Content-CNN用来学习所有视差值的概率分布,这种策略不需要考虑训练样本数量的不平衡。尽管此类数据驱动的相似度指标表现优于传统的手工指标,后处理对于获得理想的匹配结果仍然是必要的。

正则学习:基于视差图通常是分段平滑的事实,一些模型在学习过程中强制加入了平滑约束。Menze等人\cite{Menze_2015_CVPR}应用基于纹理和边缘信息的自适应平滑约束来进行立体估计。通过寻找局部不一致的标记像素,Gidaris等人\cite{gidaris2016detect}提出了检测、替换和精炼(detect, replace, refine, DRR)框架。此外,视差也可以通过结合中高级的视觉任务来实现正则化。比如,视差估计可以和语义分割同步完成\cite{yamaguchi2014};Guney和Geiger提出了Displets\cite{guney2015displets},利用目标识别和语义分割来寻找立体匹配点。

端到端的视差学习:通过仔细地设计和监督训练神经网络模型,可以直接从输入的立体图像对中学习匹配规律,从而获得比较精细的视差结果。Mayer\cite{mayer2016large}等人提出了一种新颖的网络模型DispNet,使用合成的立体图像对训练端到端的卷积神经网络。GC-Net\cite{kendall2017end}在神经网络中显式地学习特征提取、代价集(cost volume)和正则化函数。级联残差学习(CRL)\cite{pang2017cascade}是一个级联的CNN架构,采用了由粗到细和残差学习原理。

%------------------------------------------------------------------------------------
\section{本文主要内容及结构安排}
\subsection{本论文的主要内容}

\subsection{本论文的结构安排}














% 摄像机模型与标定
% 第二章 摄像机模型与标定

\chapter{摄像机模型与标定}
%------------------------------------------------------------------------------------
\section{摄像机模型}
\subsection{常用坐标系定义及变换}
% ref: 吴哲岑ch2.2,p10; 王德海ch2.1,p9; 王莹ch2.2, p7
% 图像坐标系,摄像机坐标系,世界坐标系
% 图像像素坐标系,图像物理坐标系(成像平面坐标系)
摄像机的成像变换过程涉及到三维的真实世界与二维的图像平面。在视觉测量过程中,世界空间中的点被投影到图像平面上,因此需要定义不同的坐标系并掌握各个坐标系之间的变换方法。常用坐标系包括图像坐标系、成像平面坐标系(也叫图像物理坐标系)、摄像机坐标系和世界坐标系和。

(1)图像坐标系

摄像机采集的图像是以二维(灰度图)或三维(彩色图像)数组的形式储存在计算机中的。设一张图像的分辨率为$M\times N$,则相应得到一个$M\times N$的二维数组,数组中每个元素对应图像中一个像素点。 如图\ref{fig:2_1_image_coord}所示,以图像的左上角为坐标系原点,横轴U指向右、纵轴V指向下,建立直角坐标系$O-UV$,称之为图像坐标系(或图像像素坐标系)。坐标点(u, v)对应图像中第v行第u列的像素。

\begin{figure}[!htb] %图像坐标系和成像平面坐标系
	\centering
	\includegraphics[width=2in]{figures/2_1_image_coord}
	\caption{图像坐标系和成像平面坐标系}\label{fig:2_1_image_coord}
\end{figure}

(2)成像平面坐标系

图像坐标系以像素为基本单位,而现实中以毫米等物理单位来描述物体的尺寸和位置。因此,定义成像平面坐标系$O_1-XY$来建立像素与物理长度之间的联系。在$O_1-XY$坐标系中,定义图像平面与摄像机光轴的交点$O_1$为坐标原点,也称为图像主点。理想情况下,主点位于图像的中心位置,但现实中制造的摄像机会有一定的偏差。坐标轴X、Y轴分别平行于坐标轴U、V。假设主点$O_1$在$O-UV$坐标系中的坐标为$(u_0, v_0)$,每个像素在图像宽和高方向的物理尺寸分别为$d_x$和$d_y$,则图像中某点在成像平面坐标系下的坐标$(x, y)$和图像坐标系下的坐标$(u,v )$之间的关系可表示为:

\begin{equation}\label{eq:2_1_image_coord}
\left\{
	\begin{aligned}
	u&=\frac{x}{d_x}+u_0 \\
	v&=\frac{y}{d_y}+v_0 \\
	\end{aligned}
\right.
\end{equation}

为了便于后面的坐标转换,将其表示成齐次坐标的矩阵形式:
\begin{eqnarray}\label{eq:2_1_image_coord_homogeneous}
\left[\begin{array}{c} u \\ v \\ 1 \end{array} \right]
=
\left[\begin{array}{ccc}
\frac{1}{d_x} & 0 & u_0 \\
0 & \frac{1}{d_y} & v_0 \\
0 & 0 & 1
\end{array}
\right]
\hspace{-6pt} %默认的间距有点大,不知道怎么调整。。
\left[\begin{array}{c} x \\ y \\ 1 \end{array}\right]
\end{eqnarray}

(3)摄像机坐标系

沿摄像机光轴方向移动成像平面坐标系,使坐标原点与光心$O_c$重合,定义Z轴垂直于成像平面坐标系,即摄像机的光轴,从而得到摄像机坐标系$O_c-X_cY_cZ_c$,如图\ref{fig:2_1_camera_and_world_coord}所示。摄像机光心$O_c$到成像平面坐标系原点$O_1$的距离$O_c O_1$即为摄像机的焦距$f$。

\begin{figure}[!htb] %图像坐标系和成像平面坐标系
	\centering
	\includegraphics[width=4in]{figures/2_1_camera_and_world_coord}
	\caption{图像坐标系和成像平面坐标系}\label{fig:2_1_camera_and_world_coord}
\end{figure}

(4)世界坐标系

摄像机在三维空间中的位置通常不是固定的,因此更了更方便地描述场景中目标物体的真实位置,需建立一个唯一的世界坐标系$O_w-X_wY_wZ_w$。设场景中任一点在世界坐标系和摄像机坐标系下的坐标分别为$(X_w, Y_w, Z_w)$和$(X_c, Y_c, Z_c)$,则二者之间的转换可以表示为:

\begin{eqnarray}\label{eq:2_1_world_camera_coord_transform}
\left[\begin{array}{c} X_c \\ Y_c \\ Z_c \\ 1 \end{array} \right]
=
\left[\begin{array}{cc}
\mathbf{R} & \mathbf{t}  \\
\mathbf{0^T} & 1
\end{array}
\right]
\hspace{-6pt} %默认的间距有点大,不知道怎么调整。。
\left[\begin{array}{c} X_w \\ Y_w \\ Z_w \\ 1 \end{array}\right]
=\mathbf{M_1}
\hspace{-3pt} 
\left[\begin{array}{c} X_w \\ Y_w \\ Z_w \\ 1 \end{array}\right]
\end{eqnarray}
其中,$\mathbf{R}$和$\mathbf{t}$分别用来描述两坐标系之间的旋转和平移关系,$\mathbf{R}$为$3\times 3$的正交矩阵,$\mathbf{t}$为长度为3的向量,$\mathbf{0_t}=[0, 0, 0]$,$\mathbf{M_1}$为$4\times4$的矩阵。


%------------------------------------------------------------------------------------
\subsection{摄像机线性模型}
% ref: 尚倩ch2.1,p13; 王德海ch2.1, p11; 吴哲岑ch3.1,p19; 苏东ch2.1, p10.
摄像机的线性模型即针孔(pinhole)模型\cite{zhang1999flexible, zhang2000flexible}。在此模型中,来自某一点的光线从物体上发射过来,通过一个平面上的针孔,投射在成像平面上,其余的光束都被针孔平面所阻挡,如图\ref{fig:2_1_pinhole_model}所示。因此物体在图像中的大小只需要一个参数来描述:焦距(focal length),即针孔到成像平面的距离。 设摄像机焦距为$f$,相机到所拍物体的距离为$Z_c$,所拍物体的长度为$X_c$,则利用三角形相似原理可以得到物体在图像上的长度$x$:

\begin{equation}\label{eq:2_1_triangular_similarity}
-x = f \cdot \frac{X_c}{Z_c}
\end{equation}

\begin{figure}[!htb] %摄像机针孔模型
	\centering
	\includegraphics[width=4in]{figures/2_1_pinhole_model}
	\caption{摄像机针孔模型}\label{fig:2_1_pinhole_model}
\end{figure}

负号的存在表示图像中的物体是倒立的。为了略去等式左侧的负号,我们把图像平面放到针孔的右侧,形成了更直观的三角形相似关系$x/f = X_c/Z_c$,图像中的物体由倒立变为了正立。因此,摄像机坐标系下的点$P(X_c, Y_C, Z_C)$在成像平面上的坐标可表示为

\begin{eqnarray}\label{eq:2_1_imaging_plane_and_world_transform}
Z_c
\left[\begin{array}{c} x \\ y \\ 1 \end{array} \right]
=
\left[\begin{array}{cccc}
f & 0 & 0 & 0 \\
0 & f & 0 & 0 \\
0 & 0 & 1 & 0
\end{array}
\right]
\hspace{-6pt} %默认的间距有点大,不知道怎么调整。。
\left[\begin{array}{c} X_c \\ Y_c \\ Z_c \\ 1 \end{array}\right]
\end{eqnarray}

综合式\ref{eq:2_1_image_coord_homogeneous}、\ref{eq:2_1_world_camera_coord_transform}和\ref{eq:2_1_imaging_plane_and_world_transform},可以得到空间中任意一点$P$在世界坐标系和图像坐标系下的坐标转换关系:

\begin{eqnarray}\label{eq:2_1_image_and_world_transform}
Z_c
\left[\begin{array}{c} u \\ v \\ 1 \end{array} \right]
=
\left[\begin{array}{ccc}
\frac{1}{d_x} & 0 & u_0 \\
0 & \frac{1}{d_y} & v_0 \\
0 & 0 & 1
\end{array}\right]
\hspace{-6pt}
\left[\begin{array}{cccc}
f & 0 & 0 & 0 \\
0 & f & 0 & 0 \\
0 & 0 & 1 & 0
\end{array}\right]
\hspace{-6pt}
\left[\begin{array}{cc}
\mathbf{R} & \mathbf{t}  \\
\mathbf{0^T} & 1
\end{array}\right]
\hspace{-6pt}
\left[\begin{array}{c} X_w \\ Y_w \\ Z_w \\ 1 \end{array}\right]
\\
= 
\left[\begin{array}{cccc}
f_x & 0 & u_0 & 0 \\
0 & f_y & v_0 & 0 \\
0 & 0 & 1 & 0
\end{array}\right]
\hspace{-6pt}
\left[\begin{array}{cc}
\mathbf{R} & \mathbf{t}  \\
\mathbf{0^T} & 1
\end{array}\right]
\hspace{-6pt}
\left[\begin{array}{c} X_w \\ Y_w \\ Z_w \\ 1 \end{array}\right]
=
M_1 M_2
\left[\begin{array}{c} X_w \\ Y_w \\ Z_w \\ 1 \end{array}\right]
= M
\left[\begin{array}{c} X_w \\ Y_w \\ Z_w \\ 1 \end{array}\right]
\end{eqnarray}
其中,$f_x=\frac{f}{d_x}$,$f_y=\frac{f}{d_y}$,$f_x, f_y, u_0, v_0$只与摄像机的内部结构有关,为摄像机的内部参数;$\mathbf{R}, \mathbf{t}$描述的是摄像机坐标系与世界坐标系的相对位置关系,为摄像机的外部参数。$M_1$和$M_2$分别称为内参矩阵(intrinsics matrix)和外参矩阵(extrinsics matrix)。对于一部摄像机来说,内部参数是固定不变的,而外部参数可能改变。$M=M_1 M_2$为$3\times 4$的投影矩阵,摄像机将将三维空间中的点映射到二维投影平面上的过程叫作投影变换。

注意到内参$f_x$实际上是透镜的物理焦距长度$f$与成像尺度因子$s_x=\frac{1}{d_x}$的乘积,这样做的意义在于$f$的单位是mm,$s_x$的单位是像素/mm,从而$f_x$的单位是像素。摄像机的标定并不能得到$f$和$s_x$,只有组合量$f_x$和$f_y$可以通过标定直接计算出来。

%------------------------------------------------------------------------------------
\subsection{摄像机非线性模型}
% ref: 白鹏,尚倩,王德海,吴哲岑.
针孔模型只是一个理想的模型,而实际情况下通过光心的光线太少会导致曝光不足,图像生成缓慢。因此我们生活中使用的摄像机都利用透镜来使足够多的光线收敛聚焦到成像平面上\cite{GaryBradski2009学习}。理论上可以定义不造成任何畸变的透镜,但现实中由于制造工艺问题,所有透镜都存在一定程度的畸变。为了获得较为精确的摄像机模型,需要对畸变进行建模。

摄像机的畸变主要有径向畸变(radial distortions)\cite{hartley2003multiple}、离心畸变(decentering distortions)\cite{ricolfe2010lens}和薄棱镜畸变(thin prism distortions)\cite{weng1992camera}。第一种主要是由棱镜的形状缺陷导致的,其只会造成径向位置误差;后两种主要是由棱镜和相机的不当安装导致的,其同时会造成径向和切向的位置误差\cite{weng1992camera}。

我们可以使用一组数学表达式来表示畸变,在摄像机的线性模型基础上加入畸变的表达式,得到一个摄像机的非线性模型。非线性畸变可表示为式\ref{eq:2_1_lens_distortion}的形式。

\begin{equation}\label{eq:2_1_lens_distortion}
\left\{
\begin{aligned}
x' &= x + \sigma_x (x,y)  \\
y' &= y + \sigma_y(x,y)
\end{aligned}
\right.
\end{equation}
其中,$(x,y)$是线性模型计算得到的成像平面坐标系下的坐标,$(x',y')$是计入畸变后更准确的坐标,$\sigma_x$和$\sigma_y$是非线性模型中的畸变参数。

(1)径向畸变

径向畸变又称径向失真,主要是由棱镜的径向曲率曲线的缺陷造成的。径向畸变可分为两类,符号为负的径向位移被称为桶形畸变(barrel distortion),它会使距离光心远的点挤向一起;符号为正的径向位移被称为枕形畸变(pincushion),它会使距离光心远的点更加分散。这种畸变关于光轴是严格对称的,我们可以用泰勒展开式来对径向畸变建模,如式\ref{eq:2_1_radial_distortion}。试验表明,使用$k1$和$k2$两项即可校正90\%以上的径向畸变\cite{hartley2003multiple}。

\begin{equation}\label{eq:2_1_radial_distortion}
\left\{
\begin{aligned}
\sigma_x(x,y) &= x(k_1 r^2 + k_2 r^4 + k_3 r^6 + \cdots)  \\
\sigma_y(x,y) &= y(k_1 r^2 + k_2 r^4 + k_3 r^6 + \cdots) 
\end{aligned}
\right.
\end{equation}

\begin{figure}[!htb] %桶形畸变和枕形畸变
	\centering
	\includegraphics[width=4in]{figures/2_1_radial_distortion}
	\caption{桶形畸变和枕形畸变系}\label{fig:2_1_radial_distortion}
\end{figure}

(2)离心畸变

离心畸变是镜头各部件的光心不严格共线造成的,现实中的光学系统都存在不同程度的离心畸变。这种畸变可描述为式\ref{eq:2_1_decentering_distortion}。

\begin{equation}\label{eq:2_1_decentering_distortion}
\left\{
\begin{aligned}
\sigma_x(x,y) &= 2p_2 xy + p_1 (3x^2 + y^2)  \\
\sigma_y(x,y) &= 2p_1 xy + p_2 (x^2+ 3y^2)
\end{aligned}
\right.
\end{equation}

(3)薄棱镜畸变

薄棱镜畸变是由透镜的设计、生产和相机装配中的缺陷造成的,例如透镜光轴与摄像机的面阵平面之间存在微小的倾斜。这种畸变可使用式\ref{eq:2_1_thin_prism_distortion}来表示。

\begin{equation}\label{eq:2_1_thin_prism_distortion}
\left\{
\begin{aligned}
\sigma_x(x,y) &= s_1 (x^2 + y^2) \\
\sigma_y(x,y) &= s_2 (x^2 + y^2)
\end{aligned}
\right.
\end{equation}

三种畸变中,径向畸变是最主要的畸变成分,薄棱镜畸变的影响很小,可忽略不计。因此,综合径向畸变和离心畸变,透镜畸变模型可表示为:

\begin{equation}\label{eq:2_1_radial_distortion}
\left\{
\begin{aligned}
\sigma_x(x,y) &= x(k_1 r^2 + k_2 r^4) + (2p_2 xy + p_1 (3x^2 + y^2))  \\
\sigma_y(x,y) &= y(k_1 r^2 + k_2 r^4)  + (2p_1 xy + p_2 (x^2 + 3y^2))
\end{aligned}
\right.
\end{equation}


%------------------------------------------------------------------------------------
\section{摄像机标定}
摄像机标定就是确定其内部参数和外部参数的过程。
\subsection{单目摄像机标定}
\subsection{双目摄像机标定}

\section{双目立体视觉原理}

\section{实验结果}

\section{本章小结}


% 基于YOLO的目标检测
% 第三章 基于YOLO的目标检测

\chapter{基于YOLO的目标检测}
目标检测是指从图像或视频中检测出某种或多种目标物体,并定位出目标在图像中的位置的过程。
准确、高效地检测到感兴趣的目标,是实现目标定位的前提条件。
2014年Girshick提出的R-CNN可以被视为使用深度学习进行目标检测的开山之作,其使检测效果获得了大幅的提升。
在那之后,涌现出了很多基于卷积神经网络的目标检测算法,YOLO便是其中的代表之一。
YOLO将目标检测任务转化为一个回归问题来处理,获得了实时的检测速度,因而非常适合本文的应用场景。
本章首先介绍卷积神经网络的基础知识,之后建立基于YOLO的目标检测模型。

\section{卷积神经网络介绍}
卷积神经网络(convolutional neural network, CNN)\cite{lecun1989}是一种专门用来处理具有类似网格结构的数据(如图像)的神经网络\cite{Goodfellow-et-al-2016}。其模拟了生物视觉神经的结构,对传统的神经网络做了一定的改变:将相邻两层神经元之间的全连接改为局部连接,并且同一层内的神经元共享权值。这些改变大大减少了神经网络的参数数量,降低了模型的复杂度,使得网络变得更容易训练。这种网络结构能够直接使用图像作为网络输入,避免了传统算法中人工提取特征的过程,在目标识别、图像分割、目标分类等很多任务中表现突出,得到了学术界与工业界的广泛关注和研究,已被大量应用于计算机视觉的各个领域中。


%\subsection{人工神经网络}
%\subsection{深度学习}
%\subsection{卷积神经网络的历史}

%------------------------------------------------------------------------------------
\subsection{卷积运算及其基本思想}
\subsubsection{卷积运算}
% ref: 吴文正,ch2.3.1,p18
数学上卷积运算包括连续卷积运算和离散卷积运算,它们的定义分别为:
%
\begin{equation}\label{eq:3_1_0}
y(t) = (x*w)(t) = \int_{-\infty}^{\infty}x(a)w(t-a)da
\end{equation}
\begin{equation}
y(n) = (x*w)(n) = \sum_{-\infty}^{\infty}x(i)h(n-i)
\end{equation}
其中的参数$x$是输入,参数$w$被称为核函数(kernel function)。星号表示卷积运算。注意到公式中将核函数进行了翻转,即当输入x的索引增大时,核w的索引是在减小的。通过翻转,卷积获得了可交换的性质,即$x*w=w*x$。

卷积神经网络中的卷积操作是对二维数组进行的,因此是二维离散卷积。而与常规的卷积不同的是,卷积神经网络的卷积计算中并没有进行翻转:
%
\begin{equation}
S(i,j) = (I*K)(i,j) = \sum\limits_{m} \sum\limits_{n} I(i+m,j+n)K(m,n)
\end{equation}
这种计算也被称为互相关函数(cross correlation),卷积核也被称为滤波器(filter)。

%------------------------------------------------------------------------------------
\subsubsection{卷积运算的基本思想}
% DL book, ch9.2, p313.
% 何鹏举,ch2.2,p16; 黄咨,ch3.3,p48.
% http://blog.csdn.net/zlrai5895/article/details/78634800?locationNum=6&fps=1
神经网络在过去几十年经历了多次起起落落,是因为其应用存在着很多困难。当网络较浅时,其无法刻画复杂的非线性关系,因此对于较多的数据和复杂的模式,无法得到理想的分类或预测性能;而如果网络很深或求解问题规模较大,则网络内的参数数量会急剧增加,运算量也会成指数级增长,导致训练无法完成。
卷积神经网络将卷积运算引入神经网络,而卷积运算包含了三个改进神经网络算法的重要思想:稀疏连接、参数共享和等变表示。

(1) 稀疏连接(局部感知)

传统神经网络使用矩阵乘法来建立各层网络的输入与输出之间的关系。参数矩阵中的每个元素都描述了上一层网络中某个神经元的输出与下一层网络的某个神经元的输入之间的交互,即每个输出单元与每个输入单元都存在着联系。然而,卷积网络具有稀疏连接的特点。输入图像中可能包含几万甚至几百万的像素点,但并没有必要让每个神经元都对整幅图像进行感知,因为局部区域内的像素联系比较大,而距离远的像素之间通常并没有很强的相关性。
使用只包含几十到几百个像素的小尺寸的卷积核,就可以检测图像中局部的特征,如边缘等。稀疏连接可以大大减少模型中的参数数量。

\begin{figure}[htb] %稀疏连接与全连接的对比;更深层的神经元具有更大的接受域
	\centering
	\begin{minipage}[c]{0.48\textwidth}
		\centering
		\includegraphics[width=3in]{figures/3_1_稀疏连接与全连接的对比}
		\caption{稀疏连接与全连接的对比}
	\end{minipage}
	\hfill
	\begin{minipage}[c]{0.48\textwidth}
		\centering
		\includegraphics[width=3in]{figures/3_1_更深层的神经元具有更大的接受域}
		\caption{更深层的神经元具有更大的接受域}
	\end{minipage}
\end{figure}

局部连接虽然使得每个神经元只与其前一层中很少的神经元有联系,但由于卷积神经网络有多层,通过层层的传递,处于网络中更深层的神经元实际上与相当多的浅层神经元之间存在着联系,它们比处在浅层的神经元具有更大的接受域。也就是说,深层的神经元将浅层神经元中的局部信息逐渐综合起来,从而得到了全局信息。这种模式非常类似于人类由局部到全局的视觉感知过程,因此稀疏连接的特性也被称为局部感知。

(2)参数共享

参数共享是指对于不同的神经元使用相同的权重参数。在传统神经网络中,每层网络的权重矩阵W中的元素都仅仅被使用了一次,这是非常低效的。而在卷积神经网络中,卷积层通过平移卷积核遍历了整个特征平面,也就是说每个卷积核内的参数都是在这一整层网络中共享的。卷积参数的共享使得我们不需要区别对待每个位置,只要学习一个参数集合即可。事实上,对卷积核参数的学习可以理解为对特征提取方式的学习,该方式是与位置无关的。比如当一个卷积核能用来提取沿某个方向的梯度特征时,使用该卷积核能在整个特征平面上提取到所有沿该方向的梯度特征。与稀疏连接类似,参数共享也能够显著降低网络模型中的参数数量。

\begin{figure}[htb] %卷积运算的参数共享
	\centering
	\includegraphics[width=3in]{figures/3_1_参数共享1}
	\includegraphics[width=3in]{figures/3_1_参数共享2}
	\caption{卷积核的参数共享} \label{fig:3_1_卷积核的参数共享}
\end{figure}

需要说明的是,一个二维的卷积核只能提取某种特定的特征,而图像中含有多种特征,显然只提取一种特征是非常不充分的。因此卷积层的输入输出一般是多个特征图堆叠起来的3维特征卷(feature volume),此时对应每个输出特征图的卷积核是3维的,或者认为是多个2维的卷积核,每个作用于输入特征卷中的一个特征图,则这些2维的卷积核之间是不共享参数的,如图所示\ref{fig:3_1_三维卷积核}。

\begin{figure}[htb] %三维卷积核
	\centering
	\includegraphics[width=4in]{figures/3_1_三维卷积核}
	\caption{三维卷积核} \label{fig:3_1_三维卷积核}
\end{figure}

%如图\ref{fig:3_1_卷积核的参数共享},输入图像的尺寸是5*5,使用大小为3*3的卷积核来提取图像中的特征,仅需要9个参数。而如果不共享卷积核的参数,则输出的特征图上的每个元素都需要9个参数,输出共有9个元素,故该卷积层需要9*9=81个参数。实际中的图像尺寸远大于9*9,但无论图像多大,卷积核都可以使用3*3的尺寸,则参数共享减少的网络参数数量会更为显著。

我们使用一个具体的例子来定量地分析一下稀疏连接和参数共享在减少网络模型参数数量方面的作用。对于一张尺寸为1000 * 1000的灰度图像,假设输出10个尺寸为500 * 500的特征图,当使用全连接时,该层网络参数共有1000 * 1000 * 500 * 500 * 10 = 2.5e12,如果每个参数占4字节,那么仅保存这些参数就需要9313G的空间,显然这么多的参数是无法训练的。而如果使用尺寸为3*3的卷积核,仅使用稀疏连接时,该层所需参数数量为500 * 500 * 3 * 3 * 10 = 2.25e6,相比2.5e12减小了6个数量级。再加入参数共享,则只需要3 * 3 * 10 = 90个参数。可见,稀疏连接和参数共享在减少网络模型参数数量上的作用是十分惊人的。

(3)等变表示

参数共享使得卷积层具有对平移等变(equivariance)的性质。
函数的等变性指的是函数的输入与输出以相同的方式发生改变,而卷积对平移等变也就是说,对于一张特征图,先对其进行卷积,然后对卷积结果进行平移,与先进行平移再应用卷积的效果是一样的。这也就说明了参数共享能使卷积操作对于整个特征图中的各个位置发挥相同的作用,因而一个特定的卷积核能用于提取散落在特征图中各处的相同特征。卷积对于缩放或旋转等变换并不具有等变性,但通过一些其他机制可以使其具有等变性。
比如,将旋转角度离散化,用多个卷积核各对应一个角度,则当输入旋转某个角度时,总会有一个过滤器被激活,则再使用一个最大池化单元,即可获得对旋转变换的不变性。

%------------------------------------------------------------------------------------
\subsection{卷积神经网络结构}
% ref: 吴正文,ch2.2,p18; 何鹏举,ch2.3,p19; 陈拓,ch3.1.3,p36.
卷积神经网络将传统神经网络的全连接层改为了卷积层,因此其基本网络结构一般包含输入层、卷积层、下采样层和输出层几个部分,有些网络仍然包含全连接层,如图\ref{fig:3_1_卷积神经网络结构}所示。

\begin{figure}[htb] %卷积神经网络结构
	\centering
	\includegraphics[width=4in]{figures/3_1_卷积神经网络结构}
	\caption{卷积神经网络结构} \label{fig:3_1_卷积神经网络结构}
\end{figure}

输入层:输入层的输入一般为经过某些预处理后的图像,预处理包括灰度化、像素值归一化、裁剪和调整为特定尺寸等。如果把处理后的图像也视为特征图的话,则对于灰度图,输入层为单个特征图,对于彩色图像,输入层为多个特征图,每个对应图像的一个通道。

卷积层:卷积层是卷积神经网络的核心组成部分,其作用是提取图像中的特征,因此也被称为特征提取层。通常卷积层中会使用多个卷积核以提取不同类型的特征,基于每个卷积核可以得到一个特征图(或叫特征平面),故卷积层的输入和输出都是多个特征图堆叠形成的特征卷。卷积层的每个神经元从前一层的局部感受野中提取某种特征,接近输入层的卷积层提取的特征较为低级,在经过多个卷积层之后,特征会逐渐变得更高级、更抽象,同时也会包含更大范围的信息。在对输入的特征图进行卷积运算后,还要像传统神经网络一样加入偏置项(bias),并经过某种类型的非线性激活函数以获得非线性的拟合能力:
%
\begin{equation}
y_l = f(W*y_{l-1}+b)
\end{equation}
$f(\cdot)$表示激活函数。激活函数通常具有非线性、可微性、单调性的性质,常见的激活函数有sigmoid, tanh, 整流线性单元(ReLU)等。sigmoid能把实数范围内的输入压缩到0到1之间输出,但它有一个严重的缺点,当输入非常大或非常小时,函数会饱和,梯度趋向于0,无法实现有效的梯度下降。tanh是sigmoid的变形,tanh(x) = 2sigmoid(2x) - 1,它通常比sigmoid的表现更好,因为其均值为零。整流线性单元不存在饱和的问题,当其处于激活状态时,一阶导数处处为1,因此它的梯度方向效率较高。但它的问题是一旦取值为零,则它的梯度永远为零,也就无法再次激活了,可以认为这个神经元“死亡”了。当设置的学习率较大时,网络中会有大量的神经元死亡。为了解决这个问题,出现了一些ReLU的变种,比如渗透整流线性单元(leaky ReLU)\cite{maas2013rectifier},其定义为f(x,a)=max(0,x)+a*min(x,0),a一般取为类似0.01的小值。
% leaky relu的介绍
\begin{figure}[htb] %激活函数
	\centering
	\includegraphics[width=4in]{figures/3_1_激活函数}
	\caption{激活函数} \label{fig:3_1_激活函数}
\end{figure}

每个卷积层的输出尺寸由输入特征图尺寸、卷积核尺寸、步长和边界处的零填充(zero padding)等参数共同决定。设输入、输出特征图分别为边长$x_{i}$和$x_{o}$的正方形,卷积核为边长k的正方形,卷积运算的步长为stride,零填充个数为pad,则输出特征图的尺寸可根据下式来计算:
%
\begin{equation}
x_o = \frac{x_i + 2 \times pad - k}{stride} + 1
\end{equation}


下采样层:下采样层也称为子采样层或池化层,该层的引入是为了在保留有用信息的同时尽量减小数据的维度,本质是一种聚合的操作。下采样的核心是池化函数,它决定了用某一位置所在矩形邻域内的哪一种统计特征作为该位置的输出。池化函数可以表示为:
\begin{equation}
O = (\sum \sum I(i,j)^P \times G(i,j))^{1/P}
\end{equation}
其中,I和O分别表示池化层的输入、输出特征图,G表示高斯核函数,P在$[1, \infty)$内取值。当P=1时,池化被称为平均池化,取各个子区域内元素的均值作为输出;而当$P\rightarrow \infty$时,使用的是最大池化,即输出每个子区域内的最大元素。
如果池化层的输入特征图尺寸为m*n,池化操作水平和竖直方向的步长为w和h,则下采样后输出的特征图尺寸为(m/w)*(n/h)。最常见的采样块尺寸为2*2。
池化操作除了能够降低参数数量外,也能够有效地防止过拟合现象的发生。此外,池化操作还能使卷积层的输出获得局部平移不变性的性质,也就是说,当对该层的输入进行微小的平移时,池化后该层的大多数输出都不会变化。
该性质很重要,因为我们通常只关心某个特征是否出现,而并不关心它的具体位置。

\begin{figure}[htb] %最大池化和平均池化
	\centering
	\includegraphics[width=4in]{figures/3_1_最大池化和平均池化}
	\caption{最大池化和平均池化} \label{fig:3_1_最大池化和平均池化}
\end{figure}
%此外,在卷积层中设置卷积的步长和零填充也可以实现下采样的功能。

全连接层:全连接层通常位于最后一个卷积或采样层与输出层之间,用于将二维特征图转变为一维特征向量。全连接层的每个神经元与前一层的每个神经元相连,因此相当于两个矩阵相乘。使用全连接层会给卷积神经网络带来一些限制,因为卷积操作使得卷积层可以接受任意尺寸的输入,而全连接层的神经元个数固定,因此图像尺寸不能随意改变,当输入不同尺寸的图像时,可能需要首先进行缩放或裁剪的处理。

输出层:输出层的设计基于网络的具体应用任务。比如对于图像识别,输出层通常会是一个分类器。常用的分类器包括多项式逻辑回归(Multinomial Logistic Regression)分类器、softmax分类器等,也可以使用一到两层的全连接层作为分类器。
% softmax分类器介绍: 何鹏程, ch2.3.3, p21.

%------------------------------------------------------------------------------------
\subsection{卷积神经网络的训练}
% SGD,反向传播算法;损失函数;正则化。学习率。
\subsubsection{反向传播算法}
% 这块太蛋疼了。。凑活一下吧。。
卷积神经网络的训练与传统人工神经网络的训练类似,主要可以分为有监督、无监督和二者结合等几种方式。本文使用的模型都基于有监督学习,即网络利用带标记的训练样本数据进行学习,训练数据的标记即为监督信号。卷积神经网络的训练本质上是一个优化问题,即定义一个目标函数并将其最小化,该函数也称为代价函数或损失函数。训练常使用基于梯度的方法,主要包括前向传播和反向传播两个阶段。
前向传播即由网络输入经过层层的卷积、池化等操作后求得网络输出的过程,比较简单;反向传播是指由网络的输出值与训练数据的真值之间的误差反向逐层计算各层权重与偏置项的梯度,并根据学习率来更新这些参数使损失函数值逐渐下降至最终收敛。下面简要介绍一下反向传播的计算\cite{bouvrie2006notes}。

% 实在是没有精力看论文推公式了,先抄一下。。
% <Notes on Convolutional Neural Networks>, http://cogprints.org/5869/1/cnn_tutorial.pdf
% http://www.hankcs.com/ml/sgd-cnn.html  (a,z是啥没说明就出现在公式里,看不懂,不用了)
% http://deeplearning.stanford.edu/wiki/index.php/反向传导算法
(1)全连接层的反向传播

令$l$表示当前层,输入层和输出层编号分别为$1$和$L$。定义当前层的输出为$x^l = f(u^l)$,其中$u^l = W^l x^{l-1} + b^l$,$f(\cdot)$表示激活函数。则由下一层传到当前层的残差为:
\begin{equation}\label{eq:3_1_残差_全连接层}
\delta^{l} = ((W^{l+1})^T \delta^{l+1} ) \circ f'(u^l)
\end{equation}
式中$\circ$表示逐元素的乘法。损失函数关于当前层的权重W和偏置b的梯度为:
%% align环境占两行太浪费了,一行挤一挤就好
%\begin{align}
%\frac{\partial E}{\partial W^l} &= x^{l-1}( \delta^l )^T \label{eq:3_1_w梯度_全连接层} \\
%\frac{\partial E}{\partial b^l}  &= \delta^l \label{eq:3_1_b梯度_全连接层}
%\end{align}
\begin{equation}
\frac{\partial E}{\partial W^l} = x^{l-1}( \delta^l )^T, \quad
\frac{\partial E}{\partial b^l}  = \delta^l
\end{equation}


(2)卷积层的反向传播

在卷积层中,每个输出的特征图结合了多个输入特征图的卷积结果。将当前层的输出表示为:
\begin{equation}
x_j^l = f \left( \sum\limits_{i \in M_j} x_i^{l-1} * k_{ij}^l + b_j^l \right)
\end{equation}
式中$M_j$表示输入特征图中的某一个。假设每个卷积层$l$后面都有一个下采样层$l+1$,下采样层中的权重都为$\beta$,则计算残差为:
\begin{equation}
\delta_j^l = \beta_j^{l+1} (f'(u_j^l) \circ up(\delta_j^{l+1})) \label{eq:3_1_残差_卷积层}
\end{equation}
其中$up(\cdot)$表示上采样操作,如果下采样的因子为n,则上采样将每个元素沿水平和竖直方向重复n次。进而可以计算损失关于偏置和卷积核权重的梯度:
%\begin{align}
%\frac{\partial E}{\partial k_{ij}^l} &= \sum\limits_{u,v} (\delta_j^l)_{uv} (p_i^{l-1})_{uv} \label{eq:3_1_w梯度_卷积层} \\
%\frac{\partial E}{\partial b^j}  &= \sum\limits_{u,v} (\delta_j^l)_{uv} \label{eq:3_1_b梯度_卷积层}
%\end{align}
\begin{equation}
\frac{\partial E}{\partial k_{ij}^l} = \sum\limits_{u,v} (\delta_j^l)_{uv} (p_i^{l-1})_{uv} , \quad
\frac{\partial E}{\partial b^j}  = \sum\limits_{u,v} (\delta_j^l)_{uv}
\end{equation}
$(p_i^{l-1})_{uv}$是$x_i^{l-1}$中与卷积核$k_{ij}^l$逐元素相乘的小块。使用MATLAB语句的表示更直观:
\begin{equation}
\frac{\partial E}{\partial k_{ij}^l} = rot180(conv2(x_i^{l-1}, rot180(\delta_j^l), 'valid'))
\end{equation}
'valid'表示卷积的填充方式,旋转180度是为了使用卷积函数来进行互相关操作。

(2)下采样层的反向传播

下采样层的输出表示为:
\begin{equation}
x_j^l = f \left( \beta_j^l down(x_j^{l-1}) + b_j^l \right)
\end{equation}
$down(\cdot)$表示下采样函数。则残差可使用下面的语句求取:
\begin{equation}
\delta_j^l = f'(u_j^l) \circ conv2(\delta_j^{l+1}, rot180(k_j^{l+1}, 'full'))
\end{equation}
定义$d_j^l = down(x_j^{l-1})$,则损失函数关于$b$和$\beta$的梯度分别为:
%\begin{align}
%\frac{\partial E}{\partial b^j}  &= \sum\limits_{u,v} (\delta_j^l)_{uv} \\
%\frac{\partial E}{\partial \beta^j} &= \sum\limits_{u,v} (\delta_j^l \circ d_j^l)_{uv}
%\end{align}
\begin{equation}
\frac{\partial E}{\partial b^j}  = \sum\limits_{u,v} (\delta_j^l)_{uv} , \quad
\frac{\partial E}{\partial \beta^j} = \sum\limits_{u,v} (\delta_j^l \circ d_j^l)_{uv}
\end{equation}

\subsubsection{随机梯度下降}
% SGD, DL book, ch5.10, p160.
% http://www.hankcs.com/ml/sgd-cnn.html
传统梯度下降使用的损失函数通常是对每个输入样本的代价的累积。对于使用n个样本的模型来说,计算每一个梯度的运算代价为O(n)。而深度神经网络的训练需要大量的样本数据,训练集规模可能为上亿的量级,而且模型中的参数也非常多,因此这种计算梯度的策略运算量巨大,非常耗时;另外由于计算机的内存和显存有限,一次性载入大量的训练数据也是不现实的。在这种背景下,随机梯度下降(stochastic gradient descent,SGD)算法应运而生。随机梯度下降的核心思想是,把梯度作为期望来对待,即使用一小部分的样本数据来做一个近似估计。对于训练的每一次迭代,仅从整个训练数据集中抽取很小的一批(minibatch)样本数据用于训练。这个小批量数据的个数m一般取值为1到几百,不论整个训练集的规模增长到多大,m通常都是不变的。算法\ref{alg:sgd}给出了随机梯度下降的具体步骤。
% DL book,ch8.3.1,p279.
\begin{algorithm}[htb]
	\caption{随机梯度下降在第$k$个训练迭代的更新}
	\label{alg:sgd}
	\begin{algorithmic}
		\REQUIRE 学习率 $\epsilon_k$
		\REQUIRE 初始参数$\theta$
		\WHILE{终止训练条件未满足}
		\STATE 从训练数据集中取出$m$个样本$\{ x^{(1)},\dots, x^{(m)}\}$ ,其中$x^{(i)}$对应目标为$y^{(i)}$。
		\STATE 计算梯度的估计值: $\hat{g} \leftarrow + 
		\frac{1}{m} \nabla_{\theta} \sum_i E(f(x^{(i)};\theta),y^{(i)})$
		\STATE 更新参数:$\theta \leftarrow \theta - \epsilon \hat{g}$
		\ENDWHILE
	\end{algorithmic}
\end{algorithm}

学习率是随机梯度下降中的一个很重要的参数。随机梯度下降使用的学习率比普通梯度下降的学习率要小很多,因为其梯度变动比较剧烈,方差较大。在实践中一般要随着迭代次数的增加而逐渐降低学习率,比如可以使用线性衰减的学习率,或者每隔一段时间将学习率减半。更好的方法是预留一部分数据用于每次迭代完成后计算目标函数的输出值,当相邻两次迭代输出的变化小于某个阈值时才减小学习率。一般获取每个批之前要将数据随机打乱,否则得到的梯度可能偏离正确的方向,导致收敛性差。

另外还有很多针对梯度下降的优化算法,如动量法、Adagrad、RMSprop、自适应矩估计(Adam)等,在此不再详细介绍。

\subsubsection{正则化}
% DLbook,ch5.2.2,p132; ch7,p225(L1, L2).
一般我们在训练集之外会使用一个测试集来检验模型的泛化能力,而模型经常会出现欠拟合和过拟合的问题。欠拟合指模型在训练集上不能获得理想的效果,而过拟合指模型在训练集上表现良好,但在测试集上表现很差,训练误差和测试误差之间存在较大的差距。通过调整模型的容量,也就是模型可以选择的函数的数目,可以控制模型对于这两种问题的倾向。根据统计学习理论,随着模型容量变大,训练与泛化误差之差的上界会逐渐增大,因此通过限制模型容量可以改善过拟合现象。

\begin{figure}[htb] %欠拟合和过拟合
	\centering
	\includegraphics[width=3.5in]{figures/3_1_欠拟合和过拟合}
	\caption{欠拟合和过拟合现象} \label{fig:3_1_欠拟合和过拟合}
\end{figure}

正则化就是用来降低泛化误差、改善过拟合现象的策略。最常见的操作是,在目标函数中加入参数范数的惩罚,从而限制模型参数的增长。对于神经网络中的参数,一般只对权重参数W做惩罚,而不考虑偏置参数b。常用的参数范数惩罚包括L2正则化和L1正则化,分别在目标函数中加入权重的2-范数和1-范数作为惩罚。


%------------------------------------------------------------------------------------

\section{YOLO目标检测模型}
% YOLO paper; 
% 杨眷玉《基于卷积神经网络的物体识别研究与实现》; 
% 郑嘉祺《基于DCNN的井下行人检测系统的研究与设计》.
YOLO(You Only Look Once)\cite{redmon2016you}是一种实时性非常好的基于卷积神经网络的目标检测模型,它的运行速度可以达到45帧/秒,而其简化版本甚至可以达到150帧/秒以上,非常适合于在实时性方面有较严格的要求的应用场景。

YOLO模型之所以能够实现如此高的运行速度,主要是因为在对目标检测任务进行建模时使用了一种新颖的思路,将目标检测作为一个回归问题来处理。其他主流的目标检测算法,如可变形部件模型(DPM)和基于卷积神经网络的R-CNN系列,大都将目标检测作为一个分类问题来处理,首先需要使用滑动窗口、区域提名(region proposal)等方法初步生成一系列可能包含待检测物体的检测框,再使用分类器来判断每个检测框中包含哪种物体,完成后还要进行很多后处理来改善结果,过程繁琐,耗时较大。

与此相对,YOLO的训练和检测完全基于一个单独的网络,它直接建立输入图像像素与检测框坐标和目标属于某个类别的概率之间的回归关系。因此,这个模型只需要一次性提取图像信息后即可预测图像中存在的全部物体及其具体位置,这也是模型被命名为“You Only Look Once”的原因。
基于其简单的网络结构,YOLO在整幅图像上进行训练,并能够直接根据训练数据来改善其检测的性能。
由于能够利用整张图像的全局信息来对每个检测框进行预测,YOLO不易发生对背景的误检,在获得优秀的运行效率的同时也保证了较高的检测精确度。

%------------------------------------------------------------------------------------
\subsection{目标检测策略与过程}
YOLO将每幅待检测图像分割为$S\times S$个方格(grid),图像中每个物体的中心部分所在的小方格负责该物体的检测,如图\ref{fig:3_2_物体中心所在方格负责检测该物体}所示,图中狗所在边界框的中心位于坐标为(5,2)的方格中,因此该方格负责检测该图中的狗。每个单元格要预测$B$个边界框(bounding box),并为每个边界框给出一个置信度,这个置信度既要反映单元格对于包含所要检测的物体的信心,也要体现单元格对于该检测框的位置和尺寸的准确度的把握,因此可以定义为:
\begin{equation}
Confidence = P(Object) \times IOU_{pred}^{truth}
\end{equation}
其中,$P(object)$用于描述第一个因素。如果单元格认为自己负责的范围内并没有待检测的物体,则P = 0,从而置信度也就为0。而如果单元格认为自己是包含某种目标物体的,则P = 1,从而Confidence = IOU。IOU就是用来描述第二个因素的,即单元格对检测框的位置准确性的信心。IOU全称为Intersection Over Union,某些文献内翻译为“交并比”,因为它的定义是模型预测的检测框Pred与训练数据中标记的准确包含该物体的边界框Truth的交集与并集的比值:
\begin{equation}
IOU_{pred}^{truth} = \frac{Pred \cap Truth}{Pred \cup Truth}
\end{equation}
图\ref{fig:3_2_IOU}给出了更直观的IOU的概念。IOU体现了检测框与真实边界框的重合度,我们希望重合度尽可能的高,如果两个矩形框完全重合则IOU为1,也就达到了最理想的效果。

\begin{figure}[htb] %物体中心所在方格负责检测该物体; IOU
	\centering
	\begin{minipage}[c]{0.48\textwidth}
		\centering
		\includegraphics[width=2in]{figures/3_2_物体中心所在方格负责检测该物体}
		\caption{物体中心所在方格负责检测该物体} \label{fig:3_2_物体中心所在方格负责检测该物体}
	\end{minipage}
	\hfill
	\begin{minipage}[c]{0.48\textwidth}
		\centering
		\includegraphics[width=2.5in]{figures/3_2_IOU}
		\caption{IOU示意图} \label{fig:3_2_IOU}
	\end{minipage}
\end{figure}

对于每个检测框有五个需要预测的参数,包括检测框的中心点坐标(x, y),检测框的宽度w和高度h,以及上面定义的置信度Confidence。检测框的中心点坐标(x, y)是相对于其所在小方格的偏移量,并以方格的边长为基准归一化到[0, 1]之间;而其宽度和高度则是以图像尺寸的宽度和高度来归一化到[0, 1]区间的。

在与每个检测框相关的5个参数之外,对于每个小方格,还需要预测C个物体类别的条件概率$P(Class_i|Object)$,C为模型能够预测的类别的数目。根据该条件概率的表示方式可知,其前提条件是小方格内包含某种物体。无论每个小方格中要预测的检测框的数量B是多少,我们只对每个小方格预测一组各个类别的条件概率。这样,在进行预测时,使用某个方格内每个物体类别的条件概率乘以该方格内每个检测框的置信度,即可得到每个检测框包含每种类别物体的置信度分数:
\begin{equation}
P(Class_i | Object) * P(Object) * IOU_{pred}^{truth} = P(Class) * IOU_{pred}^{truth}
\end{equation}
由上式可见,最终的置信度得分既反映了检测框包含某种待检测物体的概率,也反映了检测框定位该物体的效果,即检测框的位置与尺寸的精确程度。

% 检测流程
YOLO模型进行物体检测的具体过程如图\ref{fig:3_2_YOLO检测过程}所示。整个过程可以分为三个阶段:首先将输入图片缩放到网络的固定输入尺寸,即边长为448的正方形;之后进行卷积神经网络的前向传播过程,求取每个检测框包含每种物体的概率及检测框的位置,根据一个概率阈值来筛除置信度比较低的结果;最后利用非极大值抑制算法进一步筛除冗余的检测框,将精确度最高的结果作为最终检测结果输出。

\begin{figure}[htb] %YOLO检测过程
	\centering
	\includegraphics[width=5in]{figures/3_2_YOLO检测过程}
	\caption{YOLO检测过程} \label{fig:3_2_YOLO检测过程}
\end{figure}

%NMS
非极大值抑制(Non Maximum Suppression, NMS)算法是一种获取局部极大值的方法,在目标检测中经常被用来筛除冗余的候选框。该算法的具体步骤是,首先根据各个候选框的得分进行排序,然后以得分最高的候选框为基准,依次计算其余候选框与基准候选框的IOU,若IOU大于一定的阈值,则删除此候选框(将其得分改为0);再从剩余候选框中选择得分次高的候选框作为基准,遍历得分小于它的候选框并根据IOU来决定是否删除,重复此过程,直至最终任意两个候选框的IOU都小于所设定的阈值。非极大值抑制的效果如图\ref{fig:3_2_NMS}所示。

\begin{figure}[htb] %非极大值抑制
	\centering
	\includegraphics[width=5in]{figures/3_2_NMS}
	\caption{非极大值抑制} \label{fig:3_2_NMS}
\end{figure}

%------------------------------------------------------------------------------------
\subsection{网络结构}
YOLO的网络架构借鉴了用于图像分类的GoogLeNet\cite{szegedy2015going},其包含24个卷积层和2个全连接层,卷积层负责图像特征的提取,全连接层用来预测检测框的概率和坐标。与GoogLeNet的区别是,YOLO替换掉了结构比较复杂的Inception模块,改用1*1的卷积层实现降维,然后加上一个3*3的卷积层。具体的网络结构如图\ref{fig:3_2_YOLO网络结构}所示。

\begin{figure}[htb] %YOLO网络结构
	\centering
	\includegraphics[width=5in]{figures/3_2_YOLO网络结构}
	\caption{YOLO网络结构示意图} \label{fig:3_2_YOLO网络结构}
\end{figure}

根据前面的介绍,每个方格预测B个检测框以及C个类别的条件概率,每个检测框又包含5个参数,因此每个方格共对应B*5 + C个参数。在实际应用中,设置S = 7, B = 2,使用的数据集包含20个物体类别,即C = 20,从而卷积神经网络的预测输出为7 * 7 * (2 * 5 + 20) = 7 * 7 * 30的三维数组。

网络中使用的激活函数为渗透整流线性单元(leaky ReLU)f(x) = max(0, x) + 0.1*min(x, 0),但最后一层使用线性激活函数。另外,在最后一个全连接层之前使用了dropout\cite{hinton2012improving},在训练中随机丢弃50\%的神经元输出。dropout是一类比权重衰减计算开销更小的正则化方法,因为它不需要修改代价函数,而是改变网络本身的结构。只在一层中使用dropout并不会显著减小模型的有效容量,却可以有效地防止过拟合。

另外还有一个简化版本的Fast YOLO网络,仅使用了9个卷积层。 但由于YOLO的实时性能已经可以满足需求,而Fast YOLO的检测精度要略差一些,因此不考虑使用简化版。


\subsection{损失函数的定义} % 及改进}
由于目标检测任务需要预测出图像中物体的类型以及对应检测框的位置和尺寸,YOLO的损失函数包含了四部分内容,分别用来监督检测框的中心坐标、检测框的宽度和高度、每个检测框中是否包含待检测物体的置信度以及检测到的物体属于每种类别的概率。每一项都是误差平方和的形式,因为误差平方和的梯度比较简单,从而损失函数比较容易进行优化。

然而这种将多个损失项简单相加的形式存在诸多的问题,它并不能保证模型在这四个方面达到很好的平衡并最终获得理想的检测准确度。首先,在与每个方格对应的30维向量中,与检测框定位误差(包括位置和尺寸)相关的元素有8个,而与分类误差相关的有20个元素,给这两者的损失项赋予同等的重要性显然是不合理的;另外,在一张图像中,包含待检测物体的方格数量可能会远小于不包含待检测物体的方格数量,从而会有大量方格的置信度得分为零,而只有少量方格有非零的置信度,正负样本的分布非常不均衡,会导致正样本包含的信息在梯度下降中无法正常发挥作用,最终网络的损失函数难以收敛甚至会发散。

为了解决这些问题,我们可以对各个损失项进行加权来调整各项的重要程度。具体地,主动增加检测框坐标误差对应的损失函数权重系数$\lambda_{coord}$,并减小检测框不包含待检测物体时置信度得分的损失函数权重$\lambda_{noobj}$。

另外,针对检测框尺寸误差对应的损失项,使用误差平方和也是存在问题的。对于一个较大的检测框和一个较小的检测框,如果网络对它们的预测宽度的误差值相同,则二者的平方和损失是相同的;但显而易见,该误差对于较小的检测框的影响要远远大于对较大检测框的影响。比如,假设两个检测框宽度分别为10和2,网络对它们的预测误差都为1,则前者只偏离了十分之一,而后者却相差了一半。算法提出者通过改为预测检测框宽度和高度的平方根,在一定程度上改善了这种影响,如图\ref{fig:3_2_检测框尺寸平方根的敏感性曲线}所示,当绝对误差相同时,较大的检测框的平方根误差小于较小的检测框。

% 注意本页的空白。(图如果放在loss公式之前,公式被挤到下一页,前一页可能会有很大空白。)
\begin{figure}[htb] %检测框尺寸平方根的敏感性曲线. 
	\centering
	\includegraphics[width=5in]{figures/3_2_检测框尺寸平方根的敏感性曲线}
	\caption{检测框尺寸平方根的敏感性曲线} \label{fig:3_2_检测框尺寸平方根的敏感性曲线}
\end{figure}

前面已经说过,YOLO对每个方格预测多个(默认取B=2)检测框,但在训练时我们只选择与训练数据中物体标记框的IOU最大的检测框来负责该物体的预测,这样使得每个检测框专注于某种特定尺寸、长宽比或物体类型,从而训练出的模型具有更好的预测效果。

至此我们可以给出整个损失函数的数学表达式:
\begin{equation} \label{eq:3_2_loss_function}
\begin{split}
Loss = & \lambda_{coord} \sum_{i=0}^{S^2}\sum_{j=0}^B \mathbf{1}_{ij}^{obj}
			\left[(x_i - \hat{x}_i)^2 + (y_i - \hat{y}_i)^2\right] \\
& + \lambda_{coord}  \sum_{i=0}^{S^2}\sum_{j=0}^B \mathbf{1}_{ij}^{obj}
			\left[ \left(\sqrt{w_i} - \sqrt{\hat{w}_i} \right) ^2 + \left( \sqrt{h_i} - \sqrt{\hat{h}_i} \right) ^2\right] \\
& + \sum_{i=0}^{S^2}\sum_{j=0}^B \mathbf{1}_{ij}^{obj} (C_i - \hat{C}_i)^2
			+ \lambda_{noobj} \sum_{i=0}^{S^2}\sum_{j=0}^B \mathbf{1}_{ij}^{noobj} (C_i - \hat{C}_i)^2 \\
& + \sum_{i=0}^{S^2} \mathbf{1}_{i}^{obj} \sum_{c \in classes} ( p_i(c) - \hat{p}_i(c) )^2 \\
\end{split}
\end{equation}
式中,当第i个方格中包含待检测的物体时,$\mathbf{1}_{i}^{obj}$取值为1,否则为0;并且只有第i个方格预测的第j个检测框负责该物体的检测时,$\mathbf{1}_{ij}^{obj}$取为1,其余情况都为0。这两个系数的存在确保了只有包含待检测物体的方格会产生对应于物体分类误差的代价项(第四行的代价项,$p(c)$表示该物体属于c类别的概率),以及只有负责检测某个物体的检测框才会产生对应于边界框位置和尺寸误差的代价项(前两行的代价项)。另外需要强调,x、y、w、h分别是相对小方格边长和图像边长归一化后的结果,且x、y的原点是图像中物体所归属方格的左上角。


\subsection{模型改进} % 只写代价函数就不需要单读作为一小节。
% 杨眷玉
(1)代价函数的改进

对于代价函数中的第二项,也就是检测框宽度和高度误差对应的代价项,原算法使用平方根的方式只能在一定程度上改善“相同误差对不同尺寸检测框影响程度不同”这个问题。实际上我们可以利用归一化的思想来对其进行改进:
\begin{equation}
\lambda_{coord}  \sum_{i=0}^{S^2}\sum_{j=0}^B \mathbf{1}_{ij}^{obj}
\left[ \left(\frac{w_i - \hat{w}_i}{\hat{w}_i} \right) ^2 + \left( \frac{h_i - \hat{h}_i}{\hat{h}_i} \right) ^2\right]
\end{equation}
通过归一化操作,绝对误差与检测框的尺寸被联系了起来,相当于改为使用检测框尺寸的相对误差来计算代价项,从而解决了存在的问题。

% 《基于双目图像的行人检测与定位系统研究》[J],杨荣坚, ch2.2.1.
(2)分辨率提升 % 注意与实验结果小节的实验参数设置里的cell size统一!

YOLO对待检测的图像施加了非常强的空间约束,因为它将图像划分为7x7的网格,对于方格内的每个检测框最多只能预测单个物体,并且两个检测框只能预测同一类的物体。这种操作导致了YOLO对于相互靠的比较近的物体(比如两个物体的中心落在图像中同一个方格中)以及较小的目标的检测效果不甚理想。为了改善这一状况,我们去掉网络中的最后一个最大池化层,从而网络的输出尺寸变为了14x14,网格的分辨率提升了一倍。这样操作可以在尽量不改变原网络结构的情况下提高模型对相近物体和较小物体的检测能力。


\subsection{网络训练}
根据论文\cite{redmon2016you},YOLO网络需要在ImageNet\cite{russakovsky2015imagenet}包含一千个物体类别的数据集上进行预训练。由于该数据集是用于分类的,预训练时要对网络结构做出一些调整,仅保留前20个卷积层,在之后连接一个平均池化层和一个全连接层以适应分类的任务。预训练完成后,再提出这20个卷积层,连接上文介绍的YOLO网络结构中的后续网络层(即4个卷积层和2个全连接层),使用PASCAL VOC数据集进行训练。进行预训练的操作主要是由于训练数据集的限制。物体检测相关的数据集容量远远小于物体分类使用的数据集,因为其数据标注比较困难,物体分类数据集的标注只是一个类别标签,而对于物体检测,每张图像中可能有一个或多个物体,且需要对每个物体给出一个精确的边界框。PASCAL VOC数据集的图片数量级是$10^4$,这对训练一个二十多层的网络模型是远远不够的;而ImageNet数据集中有1400多万张图片,能够满足网络训练的需求。在ImageNet上训练好20个卷积层的参数后,在PASCAL VOC数据集上主要训练最后4个卷积层和2个全连接层的参数,如图\ref{fig:3_2_网络训练}所示,前20个卷积层的参数不会再发生显著的改变。为了加快预训练的速度,预训练时网络输入的尺寸设置为224*224;而正式训练时为了获得更高的检测精度,改用较高的图像分辨率,网络输入尺寸被设置为448*448。

\begin{figure}[htb] %YOLO网络训练
	\centering
	\includegraphics[width=6.5in]{figures/3_2_网络训练}
	\caption{YOLO网络训练} \label{fig:3_2_网络训练}
\end{figure}

% 这段放在{网络训练}还是{实验结果}?
网络的预训练需要大约一周的时间,且ImageNet数据集需占用约1T的硬盘空间。由于GPU运算资源、存储空间和时间的限制,本文的实验直接使用网络上提供的预训练结果在PASCAL VOC训练数据集上进行训练。

% pascal voc数据集介绍要不要紧跟上段?


% 数据预处理?归一化
训练使用的图像在输入网络时需先进行归一化处理,将像素值压缩到[-1, 1]区间内:
\begin{equation}
I(x, y) = \frac{I(x, y)}{255.0} * 2.0 - 1.0
\end{equation}

% 数据增强
另外进行了一些数据增强(data augmentation)操作,以增大训练数据量,减少过拟合,从而获得更好的模型泛化效果。
具体操作可选择图像尺寸变换(resize)、图像水平翻转(horizontal flip)、图像白化(whiten)、图像切割(crop)、添加高斯噪声、对比度变换和颜色变换等。对于尺寸变换、水平翻转和图像切割操作,需对图像对应的标签文件进行修改,以保证检测框的坐标与变换后的图像一致。

尺寸变换可根据下式来计算随机的图像宽高比new\_ar和缩放因子scale。w和h是输入图像的宽度和高度,randint()生成两个参数范围内的随机整数,random()生成0到1之间的随机数,$\alpha$为可调整的参数,实验中取0.2。
% yolo-tf_persistforever/src/data/image_mp_v2.py
\begin{align} %aspect ratio
new\_ar &= \frac{w + randint(-w*\alpha, w*\alpha)}{h + randint(-h*\alpha, h*\alpha)} \\
scale &= random() * 0.4 + 0.8
\end{align}

正式训练采用加入动量项的小批量随机梯度下降算法(mini-batch stochastic gradient descent)优化器,动量系数取0.9,并在对训练参数应用了指数衰减的滑动平均(exponential moving average),衰减因子取0.9999。

%------------------------------------------------------------------------------------
\section{实验结果}
% YOLO_small.ckpt并不是预训练的结果啊。而是训练完成的结果了。那还finetune个屁啊,图敢不敢贴?

\subsection{PASCAL VOC数据集}
% 数据集具体介绍。文件夹结构,xml文件内容?
PASCAL VOC数据集来源于2005年到2012年间举办的物体分类识别和检测挑战赛。目前流行使用的是2007年和2012年的数据集,因为在2007年数据集首次扩大到20个类别,而2012年的数据集包含了2008到2011年的所有图片。

PASCAL VOC数据集的20个类别可以分为四个大类:人类、7种日常使用的交通工具(火车、轿车、公共汽车、摩托车、自行车、船只和飞机)、6种动物(牛、羊、马、猫、狗和鸟类)以及6种室内常见家具或物品(餐桌、椅子、盆栽植物、沙发、瓶子和电视或显示器)。数据集分为训练集、验证集和测试集,2007版合计9963张图片,内含24640个带有具体标记数据的物体;2012版合计17125张图片,内含27450个带标记的物体。每张图片对应的xml文件内包含物体检测任务所需的物体类别、边界框的位置和尺寸信息。该数据集图像质量好,内容丰富,标注完整,非常适合用于相关模型的训练和算法的测试。
本实验使用PASCAL VOC 2007和2012的训练数据进行模型训练,使用2007版的测试数据进行模型测试。
% 先这么写吧。。只用2007训练效果肯定更差。。
% 此处可以加图。图片样本如图...所示。

\subsection{实验环境及训练平台}
本实验使用的计算机配置见表\ref{tab:3_3_experiment_environment}。
\begin{table}[htbp]
	\centering
	\caption{实验运行环境} \label{tab:3_3_experiment_environment}
	\begin{small} %{scriptsize}
		\begin{tabular}{|c|c|}\hline
			项目  & 配置参数  \\\hline
			%---------------------------------------------------------
			CPU & Intel(R) Xeon(R) CPU E5-2630 v2 @2.60GHz, 12核 \\
			内存 & 24G \\
			显卡 & NVidia GeForce GTX TITAN X \\
			显存 & 12G \\
			CUDA & V7.5 \\
			操作系统 & Ubuntu 16.04.1 LTS \\\hline
		\end{tabular}
	\end{small} %{scriptsize}
\end{table}

% tensorflow介绍。
网络模型的搭建、训练与测试使用开源深度学习平台TensorFlow\cite{abadi2016tensorflow}。
TensorFlow是谷歌开发的第二代人工智能学习平台,前身是DistBelief,主要应用于深度学习领域。其采用数据流图(data flow graph)的形式来建立模型,图中每个节点表示一个数学运算或数据的输入/输出,每条边则表示节点间的输入输出关系,边上传输的是多维数组数据,即张量(Tensor)。模型的计算过程就是张量在数据流图中的流动过程,这种设计思路使其具有高度的灵活性,可以通过组装现有模块或定义新的操作来搭建自己所需的模型。

TensorFlow具有自动求微分的功能,非常适合于深度神经网络这类基于梯度的机器学习算法。对于其他类型的机器学习算法,TensorFlow同样能够胜任,只要能够将计算表示为数据流图的形式就可以解决。另外,TensorFlow具有高度的可移植性,可以在台式机、服务器和移动设备的CPU或GPU上运行,也能够较容易地实现分布式的部署。

\subsection{实验参数设置}
实验使用的模型参数、训练和测试参数取值见表\ref{tab:3_3_params}。
\begin{table}[htbp]
	\centering
	\caption{实验参数设置} \label{tab:3_3_params}
	\begin{small} %{scriptsize}
		\begin{tabular}{|c|c|c|c|c|c|}\hline
			模型参数                   & 取值  & 训练参数              & 取值       & 测试参数   & 取值 \\\hline
			%---------------------------------------------------------
			IMAGE\_SIZE            & 448   &  LEARNING\_RATE & 0.0001   & THRESHOLD & 0.2 \\
			CELL\_SIZE               & 14     &  DECAY\_STEPS     & 30000    & IOU\_THRESHOLD & 0.5 \\
			% TODO: 注意CELL_SIZE与上一小节模型改进中提高分辨率操作后一致
			BOX\_PER\_CELL       & 2       &  DECAY\_RATE      & 0.1         & & \\
			$\lambda_{obj}$        & 1       &  BATCH\_SIZE       & 45          & & \\
			$\lambda_{noobj}$    & 0.5    &  MAX\_ITER           & 30000    & & \\
			%TODO:MAX_ITER写1w5还是3w? 不贴图就写3w吧。
			$\lambda_{class}$     & 2       &  SUMMARY\_ITER  & 10         &  & \\
			$\lambda_{coord}$    & 5       & SAVE\_ITER           & 1000      &  & \\\hline
		\end{tabular}
	\end{small} %{scriptsize}
\end{table}

\subsection{实验结果与分析}
图\ref{fig:3_3_测试结果}展示了模型训练完成后在PASCAL VOC 2007测试集上部分图片的检测效果。 考虑到本文算法实际应用时以旋翼无人机为载体,应用场景主要为室外,因此选择的都是室外包含人、交通工具、动物等目标的图片。
测试结果显示YOLO目标检测模型具有优秀的检测能力。由于在训练和检测的过程中利用了整张图像的全局信息,相较基于区域提名(region proposal)的算法,其将背景错误归为物体的概率要低很多。
\begin{figure}[htbp] %测试结果
	\centering
	\includegraphics[width=6in]{figures/3_3_voc2007_test_result1}\vspace{.1in}
	\includegraphics[width=6in]{figures/3_3_voc2007_test_result2}
	\caption{YOLO在PASCAL VOC 2007测试集上的部分测试结果}\label{fig:3_3_测试结果}
\end{figure}

YOLO的最大优点在于运行速度。基于前文介绍的计算机配置,在GPU上每张图片的处理时间为0.031s,即32Hz;在CPU上每张图片的处理时间约为0.35s。在使用GPU计算时,有一些图像的预处理(尺寸变化、像素值归一化等)是在CPU上使用python的cv2和numpy等模块完成的,若将这些运算转移到GPU上进行,则程序的运行速度还能获得一定的提升。实时检测效果如图\ref{fig:3_3_实时检测效果}所示。
\begin{figure}[htbp] %实时检测效果
	\centering
	\includegraphics[width=3.5in]{figures/3_3_realtime_detection}
	\includegraphics[width=3.5in]{figures/3_3_realtime_detection2}
	\includegraphics[width=3.5in]{figures/3_3_realtime_detection3}
	\caption{YOLO实时检测效果}\label{fig:3_3_实时检测效果}
\end{figure}

% 缺点
% 小物体、靠近的物体;不常见的长宽比;定位误差大。基于IOU的NMS也有问题(一个bbox包含另一个时)。
在测试中,我们也发现了一些效果不佳的测试图片,如图\ref{fig:3_3_反面结果}。根据文献和实验结果,可以总结出算法的主要缺点。
由于检测框是基于网格的,其对于局部聚集和堆叠的物体检测效果不佳。通过提高分割网格的分辨率可以在一定程度上改善这种状况,但无法从本质上解决问题。
当图像的宽高比偏离1较多时,检测效果会逐渐变差。这是因为图片输入网络时都需要变换到固定尺寸,而过大的宽高比变化会导致物体严重变形,从而检测结果中的检测框精度会变差,甚至直接检测失败。
%%这条要写吗?%突然觉得增大网格分辨率可能导致结果中有更多未被剔除的bbox。
%另外,基于IOU阈值的NMS有时也会出现问题,比如当一个较大的检测框包含一个较小的检测框且其IOU低于阈值时,就无法被剔除。
YOLO的定位误差相较Faster R-CNN等算法更大,而对于本文使用双目摄像机拍摄的图片,在前景背景分明的情况下我们可以利用深度信息对检测框的位置进行调整,以获得更精确的定位结果。
\begin{figure}[htbp] %反面结果。该不该贴?。。
	\centering
	\includegraphics[width=2.5in]{figures/3_3_反面结果1} % 这张比较高,缩小一点
	\includegraphics[width=3.5in]{figures/3_3_反面结果2}
	\caption{效果不佳的检测结果}\label{fig:3_3_反面结果}
\end{figure}


% 定量的结果。be honest, or make up something?
由于本文使用目标检测主要出于为基于立体视觉的目标定位设定定位目标的目的,目前暂未进行定量的平均检测准确度(mAP)的统计。平均检测准确度、检准率(Precision)和召回率(Recall,也叫查全率)是评价目标检测算法的重要指标,我们将算法的定量测试结果和与其他算法的比较列入后续工作。
%(map什么的要介绍吗?)
% 本文进统计了目标为行人/车辆的查准率、查全率。《基于YOLO算法的车辆实时检测》


%------------------------------------------------------------------------------------
\section{本章小结}
本章主要介绍了基于卷积神经网络的YOLO目标检测系统。首先介绍了卷积神经网络的相关理论知识,包括卷积运算的三个基本思想(稀疏连接、参数共享和等变表示)、CNN的结构以及模型训练中的反向传播和梯度下降算法。之后引入YOLO目标检测模型,分析了其检测策略和过程,介绍了网络结构与损失函数的定义,并针对存在的问题提出了改进方案。使用TensorFlow平台搭建了网络模型并在PASCAL VOC数据集上进行了训练,最后给出了实验结果并分析了算法的优缺点。测试结果显示,实验中训练的模型具有较好的检测效果以及出色的实时性能。







% 基于DispNetC的立体匹配
% 第四章 基于DispNetC的立体匹配

\chapter{基于DispNetC的立体匹配}
% ref: FlowNet paper ch2, Convolutional Networks part.
2012年Krizhevsky等人的工作\cite{krizhevsky2012imagenet}展示了卷积神经网络在大规模图像分类中的良好效果,带动了应用CNN到各种计算机视觉任务中的研究工作。深度卷积神经网络较传统神经网络而言,能够学习到更为复杂的非线性关系,因此能够更好地从大量数据中提取规律。同时其通过卷积操作获得的特征也比人工设计的特征更为有效。

过去几年涌现出了大量使用CNN进行立体匹配的研究成果。KITTI\cite{Menze_2015_CVPR} Stereo Evaluation排行榜的前列已被基于卷积神经网络的算法占领,此类算法在匹配精度和运行时间方面较传统方法都展现出了很明显的优势。Fisher等人\cite{fischer2014descriptor}从通过有监督/无监督训练得到的CNN中提取特征表达,并基于欧氏距离对这些特征进行匹配。Zbontar和LeCun\cite{zbontar2016stereo}训练了一个连体(Siamese)架构的CNN,用来预测图像块之间的相似度。这类基于图像块(patch)的算法的缺点是计算量较大,无法利用全局信息,在计算完匹配代价后仍然需要进行后处理操作,因此并不能显著降低算法的运行时间。

DispNet\cite{mayer2016large}是一种网络结构较为简单、端到端(end-to-end)的立体匹配网络。该网络的运行速度很快,适合应用于无人机等对实时性有较高要求的场景。本章介绍DispNet的网络结构、训练过程并给出使用其进行立体匹配的结果。

%------------------------------------------------------------------------------------
\section{DispNet网络结构}
% ref: FlowNet paper ch3; DispNet paper ch5.
DispNet是一种端到端的立体匹配卷积神经网络模型。所谓“端到端”是指输入需要匹配的左右图,即可预测得到匹配结果,无需其他后处理操作。其借鉴了用于预测光流的FlowNet\cite{dosovitskiy2015flownet}的结构并将其应用于立体匹配领域。

在卷积神经网络中,池化对于降低训练网络的训练量是必要的,而且能够使得网络具备聚合输入图像中更大范围信息的能力。但是池化会导致分辨率下降,为了获得稠密的与输入图像具有相同分辨率的预测结果,我们需要对池化后得到的粗糙结果进行细化。因此DispNet网络的结构呈沙漏形(如图\ref{fig:4_1_dispnet_shape}),包含一个收缩部分和一个扩张部分,整个网络作为一个整体,使用反向传播算法来训练。

\begin{figure}[!htbp]
	\centering\includegraphics[width=3.5in]{figures/4_1_dispnet_shape.png}
	\caption{沙漏型的网络结构\cite{dosovitskiy2015flownet}}\label{fig:4_1_dispnet_shape}
\end{figure}

\subsubsection{收缩部分}
DispNet的收缩部分包括10个卷积层,其中前两个卷积层的卷积核尺寸分别为$7\times 7$和$5\times 5$,其余卷积层的卷积核尺寸为$3\times 3$。10个卷积层中有6个取步长为2,其余为1,故一共进行了64倍的下采样。

网络的输入为一对图片,最简单的处理方式是将两张图片直接堆叠在一起输入网络,让网络自己学习如何处理成对的图像来获取所需的深度信息。每张图片有RGB三个通道,故网络输入的厚度为6。这种只包含卷积层的网络架构是DispNet的基础架构,如图\ref{fig:4_1_DispNet}所示。图中标注的数字为“分辨率@通道数(特征图层数)”。

\begin{figure}[!htbp]
	\centering\includegraphics[width=6in]{figures/4_1_dispnet_architecture}
	\caption{DispNet网络基础架构}\label{fig:4_1_DispNet}
\end{figure}

另一种处理方式是建立两个相互独立但相同的结构来分别处理输入的左右图,之后通过一定方式将它们组合起来,如图\ref{fig:4_1_DispNetC}所示。网络首先分别从两张图片生成一些有价值的特征表达,然后在更高级别将它们联系起来。这种结构有些类似于传统的匹配模式:首先从两张图片的小块区域中提取特征,然后比较获得的特征向量。

\begin{figure}[!htbp]
	\centering\includegraphics[width=6in]{figures/4_1_dispnetc_architecture}
	\caption{DispNetC网络架构}\label{fig:4_1_DispNetC}
\end{figure}

FlowNet中引入了一个“相关层(correlation layer)”来进行两个特征图之间的比较。假设有两个多通道的特征图
$f_1, f_2: \mathbb{R} \rightarrow \mathbb{R}^c$,$w$,$h$和$c$ 
分别是他们的宽度、高度和通道数。相关层可以让网络比较$f_1$和$f_2$中的每个小块(patch)。为了说明相关层的计算方式,下面只考虑某两个小块之间的比较。取边长为$K=2k+1$的方形小块,第一张特征图中以$x_1$为中心的小块与第二张特征图中以$x_2$为中心的小块的相关量定义为:
\begin{equation}\label{eq:4_1_correlation}  % 前面不要留空行,否则行间距大。
c(\mathbf{x}_1, \mathbf{x}_2) = \sum_{\mathbf{o} \in [-k, k] \times [-k, k]} { \langle \mathbf{f}_1(\mathbf{x}_1 + \mathbf{o}), \mathbf{f}_2(\mathbf{x}_2 + \mathbf{o}) \rangle }
\end{equation}

注意到方程\ref{eq:4_1_correlation}相当于神经网络中步长为1的卷积操作,区别在于这里是用数据与数据进行卷积,而不是使用卷积核,因此并没有可训练的权重参数。

计算$c(\mathbf{x}_1, \mathbf{x}_2)$需要进行$c\cdot K^2$次乘法。而遍历两图中所有的组合方式需要进行$w^2  \cdot h^2$次相关计算,如此大的计算量将无法实现高效的训练和预测。因此出于计算量的考虑,需要限制两图中patch的最大相对位移。设最大位移为$d$,则对于每个$\mathbf{x}_1$,$\mathbf{x}_2$被限制在$\mathbf{x}_1$附近$D=2d+1$的范围内,减小了两图中patch的组合数。另外,FlowNet在相关计算中还加入了步长$s_1, s_2$来进一步减小比较的数量。

由于立体匹配使用的图片是行对齐的,并没有必要进行二维的相关计算,因此简化为水平方向的一维相关计算。取最大相对位移$d$为40个像素,由于在进行相关计算之前图像已经经过了2个步长为2的卷积层,故这里40个像素对应输入图像的160个像素,已经覆盖了足够大的视差范围。相比FlowNet中的二维相关计算,采用一维相关计算大大减小了计算量,而且较FlowNet覆盖了更大的相对位移范围,同时因为不需使用步长,获得了更精细的采样。实际操作中,每个patch取为1个像素。因为加入了相关层,我们称这种网络结构为DispNetC。

%------------------------------------------------------------------------------------
\subsubsection{扩张部分}
扩张部分的主要结构是上卷积层(upconvolutional layer),由去池化(unpooling,与池化相对)和卷积组成。这种网络层已被多次使用过\cite{zeiler2011adaptive, zeiler2014visualizing, goodfellow2014generative}。为了获得精细的预测结果,我们对特征图应用上卷积,并将其和收缩部分中对应的特征图、以及上一层的较粗糙的预测结果拼接起来。这样既保留了从更粗糙的特征图传递过来的高级信息,也保留了更低网络层的特征图提供的精细局部信息。每一次上卷积使预测结果的分辨率增加一倍,一共重复4次上卷积,最终得到的预测结果的分辨率是输入图像的四分之一,宽度和高度都是原图的一半。对网络输出的结果使用插值方法即可获得与输入图像相同的分辨率。根据\cite{dosovitskiy2015flownet}的实验,继续增加上卷积层来细化预测结果相比使用双线性插值方法并不能显著提高预测精度,但由于视差必须取整数,我们使用最近邻插值来得到输入分辨率的结果。扩张部分的结构如图\ref{fig:4_1_DispNet_expanding_part}所示。

\begin{figure}[!htbp]
	\centering\includegraphics[width=6in]{figures/4_1_dispnet_expanding_part.png}
	\caption{DispNetC扩张部分结构}\label{fig:4_1_DispNet_expanding_part}
\end{figure}

DispNet网络结构的详细参数见表\ref{tab:4_1_DispNet_architecture}。收缩部分包括conv1到conv6b。扩张部分中,上卷积层(upconvN)、卷积层(iconvN,  prN)和代价层(loss)交替出现。最终模型预测的视差图为pr1层的输出。
DispNetC与基础架构的区别存在于前三层。只需要将基础架构的前两层(conv1, conv2)平均拆分为权值共享的两部分,conv1a/b层的输入通道数减小为3。conv2a/b层经过相关层(corr)后的输出通道数为$2d+1=81$,和con2a通过卷积重定向层(conv\_redir)后的结果拼接在一起,再通过conv3a将输出通道数与基础架构中conv3a的输出进行统一,其余部分与基础架构完全一致。
% leaky relu要不要介绍???
网络收缩部分每个卷积层使用的激活函数为Leaky ReLU,扩张部分不使用激活函数。

\begin{table}[htbp] % 这两个表格是不是应该放到附录里去。算了。
	\centering
	\caption{DispNet网络结构详细参数}
	\label{tab:4_1_DispNet_architecture}
	\begin{scriptsize}   % tiny, scriptsize, footnotesize, small.
		\begin{tabular}{|l|c c c|c c|c|}\hline
			网络层 & 卷积核 & 步长 & 输入/输出通道数 & 输入分辨率 & 输出分辨率 & 该层输入 \\\hline
			%---------------------------------------------------------
			conv1    & 7x7 & 2 & 6/64                & 768x384 & 384x192 & 左右图像 \\
			conv2    & 5x5 & 2 & 64/128           & 384x192  & 192x96    & conv1 \\
			conv3a & 5x5 & 2 & 128/256         & 192x96     & 96x48      & conv2 \\
			conv3b & 3x3 & 1 & 256/256        & 96x48       & 96x48      & conv3a\\
			conv4a & 3x3 & 2 & 256/512        & 96x48        & 48x24      & conv3b \\
			conv4b & 3x3 & 1 & 512/512         & 48x24        & 48x24       & conv4a \\
			conv5a & 3x3 & 2 & 512/512         & 48x24        & 24x12        & conv4b \\
			conv5b & 3x3 & 1 & 512/512         & 24x12          & 24x12       & conv5a \\
			conv6a & 3x3 & 2 & 512/1024     & 24x12          & 12x6          & conv5b \\
			conv6b & 3x3 & 1 & 1024/1024  & 12x6             & 12x6          & conv6a \\\hline
			%---------------------------------------------------------
			pr6+loss6 & 3x3 & 1 & 1024/1 & 12x6 & 12x6 & conv6b \\\hline
			%---------------------------------------------------------
			upconv5    & 4x4 & 2 & 1024/512  & 12x6        & 24x12    & conv6b \\
			iconv5        & 3x3 & 1 & 1025/512  & 24x12      & 24x12    & upconv5+pr6+conv5b \\
			pr5+loss5  & 3x3 & 1 & 512/1         & 24x12      & 24x12    & iconv5 \\
			upconv4    & 4x4 & 2 & 512/256   & 24x12      & 48x24    & iconv5 \\
			iconv4        & 3x3 & 1 & 769/256  & 48x24     & 48x24    & upconv4+pr5+conv4b \\
			pr4+loss4 & 3x3 & 1  & 256/1       & 48x24     & 48x24    & iconv4 \\
			upconv3    & 4x4 & 2 & 256/128  & 48x24     & 96x48    & iconv4 \\
			iconv3        & 3x3 & 1 & 385/128  & 96x48    & 96x48    & upconv3+pr4+conv3b \\
			pr3+loss3 & 3x3 & 1 & 128/1        & 96x48    & 96x48    & iconv3 \\
			upconv2    & 4x4 & 2 & 128/64    & 96x48    & 192x96   & iconv3 \\
			iconv2        & 3x3 & 1 & 193/64   & 192x96   & 192x96   & upconv2+pr3+conv2 \\
			pr2+loss2  & 3x3 & 1 & 64/1        & 192x96   & 192x96    & iconv2 \\
			upconv1     & 4x4 & 2 & 64/32    & 192x96    & 384x192 & iconv2 \\
			iconv1        & 3x3 & 1 & 93/32     &384x192  & 384x192 & upconv1+pr2+conv1 \\
			pr1+loss1  & 3x3 & 1 & 32/1         & 384x192 & 384x192 & iconv1 \\\hline
			%---------------------------------------------------------
		\end{tabular}
    \end{scriptsize}
\end{table}

\begin{table}[htbp]
	\centering
	\caption{DispNetC与基础架构差异部分的详细参数}
	\label{tab:4_1_DispNetC_architecture}
	\begin{scriptsize}
		\begin{tabular}{|l|c c c|c c|c|}\hline
			网络层  & 卷积核 & 步长 & 输入/输出通道数 & 输入分辨率 & 输出分辨率 & 该层输入 \\\hline
			%---------------------------------------------------------
			conv1a             & 7x7 & 2 & 3/64                & 768x384 & 384x192 & 左图 \\
			conv1b             & 7x7 & 2 & 3/64                & 768x384 & 384x192 & 右图 \\
			conv2a             & 5x5 & 2 & 64/128           & 384x192  & 192x96    & conv1a \\
			conv2b             & 5x5 & 2 & 64/128           & 384x192  & 192x96    & conv1b \\
			conv\_redir      & 1x1 & 1 & 128/64            & 192x96    & 192x96    & conv2a \\
			corr                   & --   &  - & 128/81            & 192x96    & 192x96    & conv2a+conv2b \\
			conv3a            & 5x5 & 2 & 64+81/256    & 192x96    & 96x48      & corr+conv\_redir \\\hline
		\end{tabular}
	\end{scriptsize}
\end{table}

根据论文\cite{mayer2016large}给出的结果,DispNetC的效果优于DispNet,而运行时间上两者并没有明显差异,因此本文采用DispNetC网络结构。

%------------------------------------------------------------------------------------
\section{网络训练}
网络的训练是通过端到端的方式进行的,以成对的图像作为网络输入,使用真实视差图(ground truth)对网络预测输出的视差图进行监督。监督学习需要大量的样本数据,然而实际生活中很难获得大量场景的真实视差,这给网络模型的训练带来了很大的困难。为了解决这个矛盾,本文使用FlyingThings3D和KITTI 2个数据集进行训练。

\subsection{FlyingThings3D和KITTI数据集}
FlyingThings3D\cite{mayer2016large}是使用开源的三维动画制作软件Blender生成的场景流(scene flow)数据集,其中也包含立体RBG图像及其视差的真实值。对于每个生成的场景,直接提取每个像素的三维坐标,根据虚拟双目相机的配置参数计算出视差真值。因为渲染引擎掌握生成场景中所有点的信息,故即使对于遮挡区域也能获得其视差真值,即视差的ground truth是100\%稠密的。图像中的场景是生活中常见的各种物品在空中沿随机三维轨迹飞行。FlyingThings3D数据集是专门用来训练大型卷积神经网络的,共包含32872组训练数据和3055组测试数据。

KITTI数据集有两部分,一部分是2012年制作的\cite{Geiger2012},另一部分公布于2015年\cite{Menze_2015_CVPR}。KITTI数据集包含道路场景的双目立体视频,视频是通过在一辆汽车上安装一对标定好的相机获取的,而视差真值是利用一个三维激光扫描仪得到的。KITTI包含了真实世界中的数据,但只有场景中静止部分的数据是有效的,另外由于激光扫描仪具有一定的距离和高度的工作范围限制,数据集只能提供稀疏的视差真值。KITTI2015包含200组训练数据和200组测试数据,只有训练集提供了ground truth;本文中的实验未使用KITTI2012,因为其只提供了灰度图像。


\subsection{网络训练过程}
KITTI数据集包含真实场景的数据,但其数据量较小,且ground truth是稀疏的;而Flyingthings3D中的场景是人工合成的,但其数据量巨大且包含100\%稠密的视差真值。因此,我们首先使用FlyingThings3D来训练网络,在网络学习到从图像中预测视差的模式后,再在KITTI2015数据集上进行微调(finetune),从而获得对真实立体图像的匹配效果。

训练和测试使用的数据均来自两个数据集中提供了ground truth的部分。
在Flyingthings3D上训练时,训练集包含22890组数据,测试集包含4370组数据;在KITTI2015上训练时,分别使用前160组和后40组数据作为训练集和测试集。图像输入网络前需进行预处理:使用最近邻插值调整分辨率与网络输入一致(768x384);左右图的RGB像素值范围从$[0, 255]$映射到$[-1, 1]$;视差图像素值保持为$[0, 255]$,但需要除以图像尺寸缩放的系数,因为图像尺寸改变会导致视差值的改变。

本文使用TensorFlow\cite{abadi2016tensorflow}作为搭建、训练网络模型的软件平台。
% TODO: TensorFlow只介绍一次,如果第三章写目标检测,就不用放在这里了!
((TensorFlow是谷歌基于DistBelief开发的第二代人工智能学习系统,其采用数据流图(data flow graph)的形式来建立模型,图中每个节点表示一个数学运算或数据的输入/输出,每条边则表示节点间的输入输出关系,边上传输的是多维数组数据,即张量(Tensor)。TensorFlow具有自动求微分的功能,非常适合于深度学习等基于梯度的机器学习算法。))

% TODO: Adam有必要介绍一下公式吗?
梯度下降使用Adam(Adaptive moment estimation,自适应矩估计)优化器\cite{kingma2014adam},参数设置:$\beta_1=0.9, \beta_2=0.999$。学习率初始化为$\lambda=0.0001$,在训练迭代400k次后每200k减半。
损失函数定义为6个尺度下的损失的加权和。每个尺度下的损失定义为该尺度下的预测结果$pr_n$和视差真值$gt_n$的绝对误差和。各尺度下的视差真值是由网络输入的视差图通过最近邻插值得到的。
另外加入L2正则化项,权重取为$\alpha=0.0004$,用于防止模型过拟合,提高泛化能力。
\begin{equation}\label{eq:4_2_loss_all}
loss = \sum_{n=1}^{6}{w_n * loss_n} + \alpha ||\theta||_2^2
\end{equation}
\begin{equation}\label{eq:4_2_loss_single}
loss_n = \sum_{i=1}^{H_n}\sum_{j=1}^{W_n}{|pr_n(i, j) - gt_n(i, j)|}
\end{equation}
公式中$w_n$表示尺度$n$下的损失权重,$\alpha||\theta||_2^2$为正则化项,$W_n$和$H_n$分别为尺度$n$下图像的宽和高。

由于网络比较深,且收缩部分和扩张部分之间存在直接的连接,如果同时计入6个尺度的损失,网络中低层的梯度方向受到多个损失的影响,损失函数可能无法高效地下降。为损失权重设计一个进度表(表\ref{tab:4_2_loss_weight_schedule})可以改善这种情况:开始训练时,设置最小尺度的loss6的权重为1,其余尺度的loss权重为0;随着训练的进行,逐渐增大高分辨率尺度损失的权重,降低低分辨率尺度损失的权重。这样使得网络先学习到粗糙的表示,然后向更精细的方向发展,同时损失函数不再对中低层特征进行限制。

\begin{table}[htbp]
	\centering
	\caption{损失函数权重进度表}
	\label{tab:4_2_loss_weight_schedule}
	\begin{scriptsize}
		\begin{tabular}{|l|cccccc|}\hline
			迭代次数  & $w_1$ & $w_2$ & $w_3$ & $w_4$ & $w_5$ & $w_6$ \\\hline
			%---------------------------------------------------------
			(0, 50k]           & 0 & 0 & 0 & 0 & 0.2 & 1 \\
			(50k, 100k]     & 0 & 0 & 0 & 0.2 & 1 & 0.5 \\
			(100k, 150k]    & 0 & 0 & 0.2 & 1 & 0.5 & 0 \\
			(150k, 200k]    & 0 & 0.2 & 1 & 0.5 & 0 & 0 \\
			(200k, 250k]    & 0.2 & 1 & 0.5 & 0 & 0 & 0  \\
			(250k, 300k]    & 1 & 0.5 & 0 & 0 & 0 & 0  \\
			(300k, --)          & 1 & 0 & 0 & 0 & 0 & 0 \\\hline
		\end{tabular}
	\end{scriptsize}
\end{table}


\subsection{网络测试过程}
介绍一下前向传播过程和对输出做resize的操作?

%------------------------------------------------------------------------------------
\section{实验结果}
计算机配置。

%------------------------------------------------------------------------------------
\section{本章小结}



% 旋翼无人机目标检测定位系统
% 第五章 旋翼无人机目标检测定位系统

\chapter{旋翼无人机目标检测定位系统}
在研究并实现了目标检测与立体匹配的算法后,本章将二者进行融合以实现基于立体视觉的目标定位功能,并搭建旋翼无人机目标检测定位系统。首先介绍旋翼无人机系统的硬件组成及算法集成方案,之后给出目标定位的实现流程,包括基于GrabCut算法的目标轮廓提取和中心点确定,以及三角测量后的坐标变换。最后通过真实场景的目标检测与定位实验验证了本文算法的有效性。

\section{旋翼无人机系统搭建}
\subsection{硬件系统集成}
在综合考虑了旋翼无人机的尺寸、载重能力、续航时间等因素,并比较了市场上可选择的双目摄像机和便携计算设备
产品后,本文在实验中搭建了以ZED双目摄像机为立体图像传感器、以英伟达Jetson TX2开发板为机载处理器、以Pixhawk为多旋翼飞行控制器、以组装六旋翼为飞行平台的旋翼无人机目标检测与定位系统,如图\ref{fig:5_1_旋翼无人机目标检测定位系统}所示。

\begin{figure}[htb] %旋翼无人机目标检测定位系统
	\centering
	\includegraphics[width=5.5in]{figures/5_平台介绍/旋翼无人机目标检测定位系统}
	\caption{旋翼无人机目标检测定位系统}\label{fig:5_1_旋翼无人机目标检测定位系统}
\end{figure}

1、ZED双目立体相机

ZED是Stereolabs公司推出的双目立体相机。该相机获取的立体图像质量很高,因为其内部对左右两个摄像机进行了硬件同步,使用OpenCV对其进行操作时获取的是唯一的设备ID,每次从相机读取图片时获得的是左右视图拼接在一起的单幅图片,需要手动进行分割才能得到左右图像。ZED双目相机可以输出WVGA/720P/1080P/2K等多个分辨率的图片,由于使用了USB 3.0接口,当分辨率为720P时,输出帧率可达60fps,而当输出为2.2K的高分辨率图像时,仍能保持15fps的帧率。ZED相机具有较大的视角范围,最大水平视角可达110度。

Stereolabs为每一台出厂的相机提供了工业级的标定数据,可用来与摄像机标定实验的结果进行对比;还提供了一个强大的SDK,可以直接调用API来获取实时的具有较高精度的深度结果,深度范围为0.5-20m。另外还提供了修改相机参数(饱和度、对比度、亮度、曝光等)、6自由度的位置追踪、三维空间建图、SVO视频录制等功能。ZED SDK输出的深度图精度较高,但由于要检验本文的立体匹配算法,因此实验中仅使用OpenCV获取相机拍摄的原始图像,之后利用标定实验获得的参数进行立体校正,最后使用DispNetC模型获取深度结果。使用OpenCV配置相机时分辨率只有三个选择,实验中使用1280x720。ZED相机参数见表\ref{tab:5_ZED}。

\begin{table}[htb] %ZED双目立体相机参数
	\centering
	\caption{ZED双目立体相机参数}
	\label{tab:5_ZED}
	\begin{small}
%	\begin{tabular}{|c|c|}\hline
	\begin{tabular*}{\textwidth}{@{\extracolsep{\fill}}cc} \toprule[2pt]
		基线长度 & 12cm \\
		有效深度范围   & 0.5-20m \\
		分辨率@帧率 & 2208x1242@15/1080P@30/720P@60/672x376@100 \\
		视角范围 & 96° H, 54° V \\
		物理尺寸 & 175x30x33 mm \\
		重量 & 159g \\
		接口类型 & USB 3.0 \\
		供电 & 通过USB供电,5V/380mA \\
		兼容的操作系统 & Windows 7/8/10; Linux \\
		第三方支持 & ROS, unity, OpenCV, MATLAB \\ \bottomrule[2pt]
	\end{tabular*}
	\end{small}
\end{table}
%基线长度: 12cm
%有效深度范围: 0.5-20 m
%分辨率@帧率: 2208x1242@15/1920x1080@30/1280x720@60/672x376@100
%视角范围: 96° H, 54° V
%物理尺寸: 175x30x33 mm
%重量: 159g
%接口类型: USB 3.0
%供电: 通过USB供电,5V/380mA
%兼容的操作系统: Windows 7/8/10; Linux
%第三方支持: ROS, unity, OpenCV, MATLAB

%------------------------------------------------------------------------------------
2、NVIDIA Jetson TX2嵌入式平台

Jetson TX2是人工智能计算公司英伟达(NVIDIA)于2017年3月发布的一款嵌入式超级计算平台,被誉为“嵌入式领域的AI超级电脑”。其具有强大的计算能力,同时功耗和尺寸较小,因而非常适合应用于机器人、无人机、自动驾驶、智慧城市等领域。

Jetson TX2板载6核CPU处理器(4个Cortex-A57和2个Denver核心)和集成256个CUDA核心、基于Pascal架构的GPU,CPU和GPU共享8GB 128bit LPDDR4内存,内存带宽高达每秒58GB。最高支持连接6个摄像机,拍照带宽为2.5Gbps;对于4K分辨率的视频的编解码速度达到了60fps。数据存储使用高速的eMMC,存储空间为32G,另外支持外接SATA硬盘。开发板提供了丰富的外设接口,如CAN、UART、SPI、I2C等,非常便于与其他模块的交互。Jetson TX2较上一代产品TX1的计算性能提升了一倍,同时功耗比也提高了一倍,常规运行模态下功耗只有不到7.5W,直接通过一个稳压模块连接到旋翼无人机使用的同种电池上,即可解决开发板的供电问题。

开发板上运行的操作系统是基于aarch64架构的Ubuntu 16.04,因此在台式计算机上配置的开发环境和部署的代码都可以比较容易地移植到TX2上以应用于移动平台。此外,英伟达公司为Jetson系列开发板提供了JetPack软件开发工具包,包含面向GPU加速计算的CUDA工具箱、面向深度学习的深度神经网络库cuDNN和面向计算机视觉的OpenCV4Tegra等工具,大大方便了相关应用程序的开发。

\begin{figure}[htbp]
	\centering
	\begin{minipage}[c]{0.3\textwidth}
		\centering
		\includegraphics[width=2in]{figures/5_平台介绍/ZED}
		\caption{ZED双目立体相机}
	\end{minipage}
	\hfill
	\begin{minipage}[c]{0.3\textwidth}
		\centering
		\includegraphics[width=1.5in]{figures/5_平台介绍/TX2}
		\caption{TX2开发板}
	\end{minipage}
	\begin{minipage}[c]{0.3\textwidth}
		\centering
		\includegraphics[width=1.2in]{figures/5_平台介绍/pixhawk}
		\caption{Pixhawk飞行控制器}
	\end{minipage}
\end{figure}

%------------------------------------------------------------------------------------
3、Pixhawk飞行控制器

Pixhawk是一款非常流行的开源自动驾驶控制器,可应用于固定翼、多旋翼等飞行器和智能车、船等移动机器人平台的控制。其由苏黎世联邦理工大学的实验室与3D Robotics和ArduPilot等团队合作设计而成,面向学术研究、个人爱好者和工业界,以较低的成本提供了很高的可用性。Pixhawk搭载了丰富的传感器,并提供了大量IO接口,能够满足应用开发需求;除了低成本高性能的STM32F4主处理器,还配备了一个协处理器用于失效保护,提高系统的可靠性。Pixhawk的具体硬件参数见表\ref{tab:5_Pixhawk}。
\begin{table}[htb] %Pixhawk硬件参数
	\centering
	\caption{Pixhawk硬件参数}
	\label{tab:5_Pixhawk}
	\begin{small}
%		\begin{tabular}{|c|c|}\hline
		\begin{tabular*}{\textwidth}{@{\extracolsep{\fill}}cc} \toprule[2pt]
			处理器 & 32位Arm Cortex M4主处理器(168Mhz/256KB RAM/2MB Flash),32位协处理器 \\
			传感器 & MPU6000等三轴加速度计、陀螺仪、磁罗盘;MS5611气压计 \\
			外设接口 & 5个UART串口,2个CAN,I2C,SPI,3.3/6.6V ADC,mini USB等 \\
			遥控输入 & Specktrum DSM/DSM2/DSM-X卫星输入,Futaba SBUS,PPM \\
			尺寸 & 长81.5mm,宽50mm,高15.5mm,重量38g \\
			其他 & 14路PWM输出;冗余的供电输入;外部安全开关、状态指示灯和蜂鸣器
			\\ \bottomrule[2pt]
		\end{tabular*}
	\end{small}
\end{table}


%------------------------------------------------------------------------------------
4、六旋翼无人机

由于需要挂载的机载处理器和双目摄像机重量较大,为了确保无人机具有足够的载荷能力,实验中自行组装了一台使用碳纤维材料机架、以无刷直流电机为动力的六旋翼飞行器。飞行器上的电机、电调、GPS、数传、遥控接收器等模块在此不做具体介绍。无人机空载时质量约为2kg,最大负载为5kg,使用6S 10000mAh的LiPo电池供电,续航时间约20分钟,可以满足实验需求。

%下一小节如果介绍软件系统集成,MissionPlanner就不要放在这。
此外,实验中使用运行Windows操作系统的联想笔记本电脑作为地面站,Mission Planner作为地面站软件,以进行起飞前的飞行控制相关参数设定、加速度计和磁罗盘等传感器的校准、飞行航线和任务规划,以及飞行中的状态监控。
地面站与遥控器共同组成地面控制系统,负责具体的飞行任务,确保飞行安全。

如图\ref{fig:5_1_数据链路}所示,整个系统中建立了ZED双目相机——Jetson TX2——Pixhawk——地面站的数据链路:ZED双目相机实时获取图像,通过USB传输给TX2开发板;TX2运行目标检测与立体匹配算法程序,计算出目标信息后通过串口通信发送给Pixhawk;最后Pixhawk利用数传将目标信息回传到地面站。

\begin{figure}[htb] %数据链路
	\centering
	\includegraphics[width=6in]{figures/5_平台介绍/数据链路}
	\caption{数据链路}\label{fig:5_1_数据链路}
\end{figure}


\subsection{软件和算法系统集成}
本文第三、四章分别使用TensorFlow搭建了目标检测和立体匹配的卷积神经网络模型,需要在应用时将两个模块的结果进行融合以完成对目标的定位。此外,对于立体匹配模块,网络的输入需为立体校正后的图像,因此需要先对ZED拍摄的立体图像进行处理。考虑到基于TensorFlow的主体算法部分是使用Python语言实现的,为了尽量保持系统内的一致性,我们使用开源计算机视觉函数库OpenCV的Python接口(即cv2模块)进行图像的获取、立体校正和其他预处理操作。在融合目标检测和立体匹配的结果得到摄像机坐标系下目标的三维坐标后,可使用pyserial模块进行串口通信。

飞行控制器Pixhawk上运行的是开源飞行控制软件ArduPilot。ArduPilot使用C++开发,支持固定翼飞行器、多旋翼飞行器和车辆等多种模型,对应多旋翼飞行器的软件模块为ArduCopter。由于使用了硬件抽象层机制,其可以运行于多个硬件平台上。目标在摄像机坐标系下的三维坐标转化到机体或世界坐标系下时需要利用飞行器的姿态与位置信息,因此坐标变换可采用两种方案:飞行控制器将惯性导航系统获得的姿态与位置信息通过串口发送给TX2,在TX2上完成坐标变换后发回给飞行控制器,或飞行控制器直接进行计算。由于坐标变换的计算量不大,且最终与地面站的通信是通过飞控的数传完成的,因此在ArduPilot中添加坐标变换的相关代码效率更高。
飞控与地面站的通信使用Mavlink通信协议。

综上所述,可以作出整个目标检测定位系统的算法集成示意图,如图\ref{fig:5_1_算法集成示意图}所示。图中坐标变换之前的部分全部在Jetson TX2开发板上执行,坐标变换和最后的通信由飞行控制器Pixhawk负责。

\begin{figure}[htb] %算法集成示意图
	\centering
	\includegraphics[width=6in]{figures/5_平台介绍/算法集成示意图}
	\caption{算法集成示意图}\label{fig:5_1_算法集成示意图}
\end{figure}

%------------------------------------------------------------------------------------
\section{目标定位}
% 第二章立体校正最后的重投影矩阵Q,《学习opencv》P503.
在使用YOLO网络模型完成了目标检测的任务后,对于每个目标,我们得到了一个将其包含在内的检测框;另一方面,利用DispNetC网络模型,我们得到了整张图片的视差图,也就获得了图像中任意像素点对应三维空间中的点的深度信息。只需要在检测框中确定对应目标中心位置的像素点,就可以根据双目立体视觉理论计算得到目标在摄像机坐标系下的三维坐标。再进行相应的坐标变换操作,即可完成目标在所需坐标系下的定位任务。

% 两个系统融合的框图

\subsection{图像坐标系下目标中心的确定}
%给两个方案:grabcut和直接取bbox中心,根据应用要求来选。
目标检测算法对每个目标返回一个矩形框,要确定目标中心在图像中的位置,最简单的方法是直接选取矩形框的中心。然而,由于目标检测算法的精度、摄像机拍摄时的姿态与目标形状等因素的影响,这种简单的方法并不能保证所定位的目标中心的准确性。因此,本文使用图像分割算法GrabCut来提取目标的轮廓,之后在轮廓范围内进行中心坐标的计算。

% 《基于双目立体视觉的目标识别与抓取定位》,王德海。ch4.1, p52.
% http://blog.sina.com.cn/s/blog_1584387c90102x5fu.html
% http://blog.csdn.net/peaceinmind/article/details/50135081
GrabCut算法\cite{rother2004grabcut}是一种基于图割(Graph Cuts)\cite{boykov2001interactive}的交互式图像分割算法。图割作为一种能量优化算法,被广泛地应用于图像分割、立体视觉、抠图等领域。在图像分割中,其将图像表示为一个无向图,每个像素作为图中的一个节点,在图中添加两个节点s和t分别代表前景和背景,在相邻像素点之间、每个像素与s/t节点之间建立边。然后构造一个能量函数,区域项由每个像素属于前景和背景的概率决定,边界项由相邻像素之间的相似程度决定,从而把图像分割问题转化为一个无向图的最小割问题。

\begin{figure}[htb] %Graph Cuts示意图
	\centering
	\includegraphics[width=4in]{figures/5_2_graphcuts}
	\caption{Graph Cuts示意图}\label{fig:5_1_Graph Cuts示意图}
\end{figure}

GrabCut对Graph Cuts算法做了一些改进,使用高斯混合模型(GMM)代替灰度直方图来分别对前景和背景进行建模,从而可以支持对彩色图像的分割;模型参数的学习是通过迭代方式进行的,而不是对能量函数进行一次性最小化;只需提供背景区域的像素集,不再需要分别指定目标和背景中的种子点。

%opencv grabcut函数介绍。
具体地,GrabCut需要用户提供一个矩形框,矩形框外的区域确定为背景,矩形框内包含前景,同时也包含部分背景。目标检测得到的检测框恰好可以用来作为这个矩形框。有时目标检测得到的检测框可能不准确,因此可以设定一个阈值,将检测框向外扩大一定的范围再传入GrabCut算法。
分割结束后会得到一个掩码矩阵mask,根据其中的元素值即可确定每个点是否属于前景。使用下式计算前景区域内像素坐标平均值,作为图像中的目标中心:
\begin{equation}
\left\{
\begin{array}{l}
\bar{u} = \frac{1}{m}\sum\limits_{i=1}^{m}u_i \\
\bar{v} = \frac{1}{m}\sum\limits_{i=1}^{m}v_i \\
\end{array}
\right.
\end{equation}
其中,$(\bar{u}, \bar{v})$为目标中心坐标,$(u_i, v_i)$为第i个像素点的坐标,m为分割结果中前景区域内的像素点数量。


GrabCut有一定的时间消耗,因此应该根据应用的实时性要求来选择确定目标中心的方法。当实时性要求高时,直接使用目标检测框的中心作为近似的目标中心;当目标为静止目标、对实时性没有严格要求时,首先使用GrabCut算法提取目标轮廓,以获得更精确的目标中心位置。

\subsection{补充坐标系定义及变换} %云台要不要?不想要
第二章中介绍了图像与摄像机相关的坐标系。由于要将双目摄像机挂载在旋翼无人机上,在此对无人机的控制与导航系统使用的坐标系进行补充。

1、机体坐标系$O_b-x_by_bz_b$

机体坐标系固连在飞行器上,原点位于其质心处。$x_b$轴位于无人机的对称平面内,水平指向前方;$y_b$轴垂直于无人机的对称平面,指向右方;$z_b$轴沿对称平面内与x轴垂直指向下方。

2、导航坐标系$O_n-x_ny_nz_n$

导航坐标系又称为地理坐标系,实际上与前面定义的世界坐标系的概念是类似的,但在定义世界坐标系时并未规定各坐标轴的方向,故在此明确为“北-东-地”(NED)坐标系。选定地球表面某一点作为坐标系原点,$x_n$轴由原点水平指向北,$y_n$轴水平指向东,$z_n$轴垂直于地面指向下方。

无人机的姿态,即机体坐标系与导航坐标系之间的关系,一般使用三个欧拉角来描述。机体坐标系的$x_b$在水平面$x_n O y_n$内的投影与导航坐标系的$x_n$轴之间的夹角被称为偏航角(yaw angle, $\psi$);机体坐标系的$x_b$轴与水平面$x_n O y_n$之间的夹角被称为俯仰角(pitch angle, $\theta$);机体轴$z_b$与$x_b$轴所在的铅垂面之间的夹角被称为滚转角(roll angle, $\phi$)。偏航角取值范围为$[0^{\circ}, 360^{\circ}]$,右偏航为正;俯仰角和滚转角的取值范围为$(-90^{\circ}, 90^{\circ})$,无人机抬头时俯仰角为正,向右滚转时滚转角为正。

机体坐标系与导航坐标系之间的转换可使用旋转矩阵(也叫方向余弦矩阵)来完成。由于旋转不具有交换性,因此由导航坐标系变换到机体坐标系需严格按照偏航角$\rightarrow$俯仰角$\rightarrow$滚转角的顺序依次进行旋转。首先绕$z_n$轴转动偏航角的旋转矩阵为:
%
\begin{eqnarray}
C_\psi =
\begin{bmatrix}
cos\psi & sin\psi & 0 \\
-sin\psi & cos\psi & 0 \\
0 & 0 & 1
\end{bmatrix}
\end{eqnarray}

其次,绕$y_n$轴转动俯仰角的旋转矩阵为:
%
\begin{eqnarray}
C_\theta =
\begin{bmatrix}
cos\theta & 0 & -sin\theta \\
0 & 1 & 0 \\
sin\theta & 0 & cos\theta
\end{bmatrix}
\end{eqnarray}

最后,绕$x_n$轴旋转滚转角的旋转矩阵为:
%
\begin{eqnarray}
C_\phi =
\begin{bmatrix}
1 & 0 & 0 \\
0 & cos\phi & sin\phi \\
0 & -sin\phi & cos\phi \\
\end{bmatrix}
\end{eqnarray}

将三个旋转矩阵结合,即可得到由导航坐标系到机体坐标系的旋转矩阵:
%
\begin{eqnarray}
C_n^b = C_\phi C_\theta C_\psi =
\begin{bmatrix}
cos\theta cos\psi & cos\theta sin\psi & -sin\theta \\
sin\theta cos\psi sin\phi - sin\psi cos\phi & sin\theta sin\psi sin\phi + cos\psi cos\phi & cos\theta sin\phi \\
sin\theta cos\psi cos\phi + sin\psi sin\phi & sin\theta sin\psi cos\phi - cos\psi sin\phi & cos\theta cos\phi
\end{bmatrix}
\end{eqnarray}

旋转矩阵为正交矩阵,从而直接可以得到由机体坐标系到导航坐标系的旋转矩阵:
%
\begin{eqnarray}
C_b^n = (C_n^b)^{-1} = (C_n^b)^T
%\begin{bmatrix}
%cos\theta cos\psi & sin\theta cos\psi sin\phi - sin\psi cos\phi & sin\theta cos\psi cos\phi + sin\psi sin\phi \\
%cos\theta sin\psi & sin\theta sin\psi sin\phi + cos\psi cos\phi & sin\theta sin\psi cos\phi - cos\psi sin\phi \\
%-sin\theta & cos\theta sin\phi & cos\theta cos\phi
%\end{bmatrix}
\end{eqnarray}

%  云台:《旋翼无人机跟踪地面移动目标的视觉控制》[D],姜运宇,P17.
另外,摄像机坐标系与机体坐标系之间也需要进行转换。考虑一般情况,相机通过云台安装在无人机上,云台具有两个旋转自由度。由于云台安装位置与无人机质心之间的距离一般应远小于无人机与目标之间的距离,因此近似认为摄像机坐标系的原点与机体系的原点重合。二者的关系如图\ref{fig:5_2_机体坐标系与摄像机坐标系的关系}所示。
$\alpha$和$\beta$分别表示云台的方向角和俯仰角。

\begin{figure}[htb] %机体坐标系与摄像机坐标系的关系
	\centering
	\includegraphics[width=3in]{figures/5_2_机体坐标系与摄像机坐标系之间的关系}
	\caption{机体系与摄像机系的关系}\label{fig:5_2_机体坐标系与摄像机坐标系的关系}
\end{figure}

从而,由机体坐标系到摄像机坐标系的旋转矩阵为\cite{姜运宇2014旋翼无人机跟踪地面移动目标的视觉控制}:
%
\begin{eqnarray}
C_b^c =
\begin{bmatrix}
-sin\beta & cos\beta & 0 \\
sin\alpha cos\beta & sin\alpha sin\beta & cos\alpha \\
cos\alpha cos\beta & cos\alpha sin\beta & -sin\alpha
\end{bmatrix}
\end{eqnarray}

由于本文实验搭建的旋翼无人机目标检测定位系统中ZED双目立体相机是直接固定在机身上的,相当于云台的方向角$\alpha$和俯仰角$\beta$都为0,因此转换矩阵简化为:
%
\begin{eqnarray}
C_b^c =
\begin{bmatrix}
0 & 1 & 0 \\
0 & 0 & 1 \\
1 & 0 & 0
\end{bmatrix}
\end{eqnarray}
由相机坐标系到机体坐标系的转换矩阵$C_c^b = (C_b^c)^{-1} = (C_b^c)^T$。

\subsection{各个坐标系下的目标定位}
在确定了目标中心在图像中的坐标(u, v)及对应视差值d后,可以直接利用三角测量原理计算目标在摄像机坐标系下的坐标。第二章中使用立体校正函数stereoRectify()得到了图像坐标系到摄像机坐标系的重投影矩阵Q,有
%
\begin{eqnarray}
\begin{bmatrix}
X \\ Y \\ Z \\ W
\end{bmatrix}
= Q
\begin{bmatrix}
u \\ v \\ d \\ 1
\end{bmatrix}
\end{eqnarray}
从而目标在摄像机坐标系下的三维坐标$(X_c, Y_c, Z_c)$ = (X/W, Y/W, Z/W)。

之后可使用坐标变换计算目标在机体坐标系下的坐标。由于忽略了摄像机坐标系与机体坐标系原点间的距离,因此只使用旋转矩阵即可得到
%
\begin{eqnarray}
\begin{bmatrix}
X_b \\ Y_b \\ Z_b
\end{bmatrix}
= C_c^b
\begin{bmatrix}
X_c \\ Y_c \\ Z_c
\end{bmatrix}
\end{eqnarray}

最后计算目标在导航坐标系下的坐标时,除了进行旋转变换,还要从旋翼无人机的惯性导航系统中读取无人机当前相对于导航坐标系原点的坐标$(X_0, Y_0, -H)$作为平移向量加入结果中:
%
\begin{eqnarray}
\begin{bmatrix}
X_n \\ Y_n \\ Z_n
\end{bmatrix}
= C_b^n
\begin{bmatrix}
X_b \\ Y_b \\ Z_b
\end{bmatrix}
+
\begin{bmatrix}
X_0 \\ Y_0 \\ -H
\end{bmatrix}
\end{eqnarray}
将结果发送回地面站时,也可以选择将惯性坐标系下的坐标转换为GPS坐标,在此不再具体介绍。最后给出完整的目标定位系统流程图,如图\ref{fig:5_2_目标定位系统流程图}。

\begin{figure}[htb] %目标定位系统流程图
	\centering
	\includegraphics[width=6in]{figures/5_2_目标定位系统流程图}
	\caption{目标定位系统流程图}\label{fig:5_2_目标定位系统流程图}
\end{figure}

%------------------------------------------------------------------------------------
\newpage
\section{实验结果}
在此对照整个算法系统流程,详细地给出每个步骤的实验结果。首先,使用OpenCV的imread()函数获取ZED拍摄的图像并进行切割,得到左右图像,如图\ref{fig:5_3_原始图像}所示。可以明显看出图像受到相机畸变的影响,地面发生了严重的扭曲。显然这样的图像是无法用于立体匹配的。
\begin{figure}[htb] %原始图像
	\centering
	\includegraphics[width=3in]{figures/5_实验结果/left_7_raw}
	\includegraphics[width=3in]{figures/5_实验结果/right_7_raw}
	\caption{原始图像}\label{fig:5_3_原始图像}
\end{figure}

使用OpenCV的stereoRectify()函数进行立体校正,得到Rl、Rr、Pl、Pr四个矩阵,然后利用InitUndistortRectifyMap()函数计算校正前后图像的逐像素映射关系矩阵。之后就可以使用remap()函数通过查表操作来实现高效的立体校正。校正结果如图\ref{fig:5_3_立体校正结果}所示,原始图像中地面上扭曲的直线已经恢复正常。
\begin{figure}[htb] %立体校正结果
	\centering
	\includegraphics[width=3in]{figures/5_实验结果/left_7_rect}
	\includegraphics[width=3in]{figures/5_实验结果/right_7_rect}
	\caption{立体校正结果}\label{fig:5_3_立体校正结果}
\end{figure}

下一步,将左图输入YOLO目标检测系统,输出目标检测结果如图\ref{fig:5_3_目标检测结果};将左右图同时输入DispNetC网络进行立体匹配,得到与左图对应的视差图,如图\ref{fig:5_3_立体匹配结果}。目标检测的结果较为理想,上下边界准确地定位到了图像中人物的头顶与脚下,左右边界略宽,但并没有偏离,目标基本位于检测框的正中央。检测框的左上角和右下角的顶点坐标分别为(665, 212)和(859,621)。
立体匹配的结果不太明显,但可以看出图中人物的大致轮廓。这主要是由于ZED双目相机的基线较短,只有12厘米,匹配结果中最大的视差也只有40左右。
\begin{figure}[htb] %目标检测结果,立体匹配结果
	\centering
	\begin{minipage}[c]{0.48\textwidth}
		\centering
		\includegraphics[width=3in]{figures/5_实验结果/left_7_detection}
		\caption{YOLO目标检测结果}\label{fig:5_3_目标检测结果}
	\end{minipage}
	\hfill
	\begin{minipage}[c]{0.48\textwidth}
		\centering
		\includegraphics[width=3in]{figures/5_实验结果/disparity}
		\caption{DispNetC立体匹配结果}\label{fig:5_3_立体匹配结果}
	\end{minipage}
\end{figure}

之后,使用OpenCV的grabCut()函数进行图像分割,提取检测框内目标的轮廓。实验中分别对RGB图像和视差图进行了分割,最终得到的目标轮廓如图\ref{fig:5_3_RGB分割结果}和\ref{fig:5_3_视差图分割结果}所示。在RGB图像上进行分割获得的目标轮廓比视差图上的结果更加准确。由于人物的脚部与地面的深度是连续变化的,因此难以在视差图中将二者分开。
\begin{figure}[htb] %目标检测结果,立体匹配结果
	\centering
	\begin{minipage}[c]{0.48\textwidth}
		\centering
		\includegraphics[width=3in]{figures/5_实验结果/rgb_seg/left_seg5}
		\caption{RGB图像分割结果}\label{fig:5_3_RGB分割结果}
	\end{minipage}
	\hfill
	\begin{minipage}[c]{0.48\textwidth}
		\centering
		\includegraphics[width=3in]{figures/5_实验结果/disparity_seg/disp_seg5}
		\caption{视差图分割结果}\label{fig:5_3_视差图分割结果}
	\end{minipage}
\end{figure}

计算分割结果中前景像素点的坐标平均值,对应RGB图像和视差图的结果分别为(754, 395)和(754, 434)。检测框的中心坐标为(762, 417)。以分割RGB图像得到的结果(754, 395)作为目标中心坐标,视差图中该位置的像素值为28,即目标中心的视差d=28。利用图像坐标系到摄像机坐标系的重投影矩阵Q,可计算
%
\begin{eqnarray}
\begin{pmatrix} X \\ Y \\ Z \\ W \end{pmatrix}=
\mathbf{Q} \begin{pmatrix} u \\ v \\ d \\ 1 \end{pmatrix}
= \begin{pmatrix} 154.51 \\ 39.27 \\ 682.57 \\ 0.23296 \end{pmatrix}
\end{eqnarray}

从而,目标在摄像机坐标系下的三维坐标为(X/W, Y/W, Z/W) = (663.25, 168.57, 2929.99)mm,计算得到摄像机坐标原点到目标中心的距离为3.01米。拍摄图片时使用手持式激光测距仪进行了测距,但考虑到测量时无法准确定位摄像机光心和目标中心,测得的距离也难免存在误差,因此需要更合理的误差分析方式。
为了尽量避免测量误差造成的影响,我们使用第二章中标定相机时采集的一组图片,定位标定板中棋盘格的四个顶点,计算所围成矩形的各边长度,与真实值进行比较,如图\ref{fig:5_3_棋盘格顶点定位}所示。实验结果见表\ref{tab:5_棋盘格顶点定位结果}和\ref{tab:5_棋盘格定位误差分析}。
\begin{figure}[htb] %棋盘格顶点定位
	\centering
	\includegraphics[width=4in]{figures/5_实验结果/chessboard_corner}
	\caption{棋盘格顶点定位}\label{fig:5_3_棋盘格顶点定位}
\end{figure}

\begin{table}[htb] %棋盘格顶点定位结果
	\centering
	\caption{棋盘格顶点定位结果}
	\label{tab:5_棋盘格顶点定位结果}
	\begin{small}
		%		\begin{tabular}{|cc|cc|cc|}\hline
		\begin{tabular*}{\textwidth}{@{\extracolsep{\fill}}cccc} \toprule[2pt]
			顶点 & 图像坐标   & 视差d & 三维坐标(mm) \\ \midrule[1pt]
			A     & (565, 481) & 25     & (-165.73, 601.95, 3279.89) \\
			B     & (672, 473) & 25     & (348.43, 563.51, 3279.89) \\
			C     & (689, 610) & 27     & (398.25, 1131.32, 3036.94) \\
			D     & (573, 620) & 27     & (-117.86, 1175.81, 3036.94) \\
			\bottomrule[2pt]
		\end{tabular*}
	\end{small}
\end{table}

\begin{table}[htb] %棋盘格定位误差分析
	\centering
	\caption{棋盘格定位误差分析}
	\label{tab:5_棋盘格定位误差分析}
	\begin{small}
		%		\begin{tabular}{|cc|cc|cc|}\hline
		\begin{tabular*}{\textwidth}{@{\extracolsep{\fill}}cccc} \toprule[2pt]
			线段 & 解算长度(mm) & 实际长度(mm) & 误差(\%) \\ \midrule[1pt]
			AB   & 515.59            & 500               & 3.12 \\
			BC   & 619.61            & 600                & 3.27 \\
			CD   & 518.03           & 500                & 3.61 \\
			AD   & 625.01           & 600                & 4.17 \\
			平均误差 &  &  & 3.54 \\
			\bottomrule[2pt]
		\end{tabular*}
	\end{small}
\end{table}

对于更远的距离,实验中测试了10m左右距离的目标,误差达到了10\%,由于其视差图中像素值都很小,因此未在文中给出。
事实上,实验中的定位精度主要受到了ZED双目相机基线长度的制约。根据三角测量原理,深度与基线长度成正比,与视差成反比,因此立体视觉系统仅在有限的距离内具有较高的精度。结合实验中使用的相机的具体参数,可知深度与视差大致满足Z=82039.66/d的关系。做出视差与深度的对应关系表格如下。由表中可见,10米距离对应的视差为8,当视差变化1,深度的变化已经超过了1米,因此即使立体匹配的结果是准确的,仅因视差的不连续性导致的误差就可达到5\%,而如果立体匹配的结果有误差,最终得到的距离误差会更大。使用更长基线的双目相机,是提高目标定位有效范围和精度的直接途径。

\begin{table}[htb] %ZED相机视差与深度的对应关系
	\centering
	\caption{ZED相机视差与深度的对应关系}
	\label{tab:5_ZED相机视差与深度的对应关系}
	\begin{small}
%		\begin{tabular}{|cc|cc|cc|}\hline
		\begin{tabular*}{\textwidth}{@{\extracolsep{\fill}}cccccc} \toprule[2pt]
			视差d & 深度Z(mm) & 视差d & 深度Z(mm) & 视差d & 深度Z(mm) \\ \midrule[1pt]
			50      & 1641          & 12      & 6836         & 6        & 13673 \\
			40      & 2051         & 11       & 7458         & 5        & 16408 \\
			30      & 2735         & 10       & 8204        & 4        & 20510 \\
			25      & 3282         & 9        & 9115         & 3        & 27346 \\
			20      & 4102         & 8        & 10255       & 2        & 41019 \\
			15      & 5469         & 7        & 11720        & 1         & 82040 \\
			\bottomrule[2pt]
		\end{tabular*}
	\end{small}
\end{table}


%------------------------------------------------------------------------------------
%\section{应用讨论?}
%\subsection{飞行控制架构}
%\subsection{智能跟随}
%\subsection{自主避障}

%------------------------------------------------------------------------------------
\section{本章小结}
本章搭建了旋翼无人机目标检测定位系统。首先介绍了硬件系统的组成,包括ZED双目立体相机、英伟达Jetson TX2开发板、Pixhawk飞行控制器和六旋翼无人机等。之后给出了集成目标检测和立体匹配模块进行目标定位的思路,利用GrabCut算法对检测结果进行分割,从而确定出图像坐标系下目标中心坐标,结合立体匹配得到的视差图,根据三角测量原理完成摄像机坐标系下的目标定位。然后补充了无人机相关坐标系的定义及转换,通过坐标变换实现机体系和导航系下的目标定位。最后给出了实验结果,并进行了误差分析。








% 总结与展望
% 第六章 总结与展望

\chapter{总结与展望}

\section{论文总结}
旋翼无人机由于其灵活的机动性与较低的成本,在生活中得到了越来越广泛的应用。对环境中各种目标的感知与交互能力是提高旋翼无人机的任务能力、保障飞行安全的重要因素。本文以基于双目立体视觉的目标检测与定位算法为研究内容,设计并实现了旋翼无人机目标检测定位系统。本文主要完成的工作如下:

1、建立了摄像机模型,并介绍了双目立体视觉理论。首先根据摄像机成像过程定义了相关坐标系,给出了坐标变换关系,分别建立了摄像机的线性和非线性模型。介绍了双目立体视觉原理中的三角测量、对极几何等理论,推导并应用 Bouguet算法和张正友标定法完成了ZED双目相机的标定实验,获得了良好的标定精度。 

2、提出了基于YOLO的改进目标检测算法。在具体分析了YOLO目标检测算法的策略与过程后,针对模型代价函数的不合理处和算法召回率较低的问题,对网络模型的代价函数和网络结构进行了修改。在TensorFlow平台上搭建了网络模型,使用PASCAL VOC数据集进行了训练,并给出了多种目标物体的实验结果。算法在GPU上的运行速度在30Hz以上,能够满足实时应用的要求。 

3、建立了DispNetC立体匹配网络模型。介绍了端到端的立体匹配模型DispNet及加入相关层的DispNetC网络结构,使用TensorFlow搭建了DispNetC网络,分别在FlyingThings3D和KITTI 2015数据集上进行了训练和微调,给出了两个数据集上的测试结果,并与SGM、SPS-St等算法进行了对比,分析了算法的优缺点。测试结果显示DispNetC具有较高的匹配精度,而运行时间仅为0.06秒,远快于大多数的立体匹配算法。最后分析了训练集大小对网络匹配精度的影响,比较结果说明增大训练数据量有利于提高匹配精度。 

4、搭建了旋翼无人机目标检测定位系统。首先介绍了硬件系统的组成,然后提出了融合目标检测与立体匹配结果进行目标定位的方案,利用GrabCut算法提取目标轮廓并确定目标中心,结合立体匹配结果,应用三角测量原理完成摄像机坐标系下的目标定位,再通过坐标变换实现机体系和导航系下的目标定位。最后给出了实验结果,并进行了误差分析。 

\vfill

\section{工作展望}
本文完成了基于双目立体视觉的目标检测与定位算法框架,并搭建了旋翼无人机目标检测定位系统。但由于时间、个人知识水平和实验条件等因素的限制,在算法研究与应用实现等方面仍存在需要改进和完善的地方。

1、目标检测方面,训练使用的PASCAL VOC数据集包含20种物体,可根据具体的应用需求对物体类别进行调整。其次,由于无人机的视角与日常的图片有一定区别,因此可考虑收集飞行中拍摄的图像并制作数据集,用来训练或finetune目标检测网络模型,以获得更好的检测效果,并增加对某些类别的支持(如树、电线杆等)。另外,算法存在的某些缺陷仍待解决。原算法作者已提出了一些提高召回率和检测框定位精度的手段,可进行研究。
%改进算法的定量分析工作和与其他算法的比较尚未完成

2、立体匹配方面,由于训练使用的数据集对应双目相机的基线长度远比实际应用时ZED双目相机的基线长,也就是说训练图像的平均视差大于实际应用场景的平均视差,因此算法迁移时精度会有所下降。双目相机的基线长度是限制其测距距离的重要因素,因此使用基线更长的相机是比较好的选择。另外,目标定位前使用图像分割算法确定目标中心的时间代价较大,可尝试在视差图上使用聚类方法提取目标轮廓。

3、旋翼无人机目标检测定位系统中有许多工作可以开展。相机可使用云台来安装在无人机上,以获得更灵活的视角;安装图传设备将目标图像实时回传,获得可视的目标定位结果;加入数据链路的反向数据流,即由地面站发送指令,实时调整目标检测的物体类别、阈值等参数。视觉信息可与无人机的控制、导航系统进行交互,以完成具体的任务。


% YOLO根据所需检测的类别进行简化;引入YOLO v2的改进;YOLO测试结果的定量分析和与其他算法的比较;
% detection和stereo correspondence针对特定应用场景的专门数据集的训练;数据集类别的增加(树、电线杆);
% 旋翼无人机系统面向应用的功能开发。

% dispnet可以引入图像边缘的信息?

% disparity和rgb分割结果怎么融合?
%换更大基线的相机?增加云台。
% 地面站部分的开发。图像显示,目标位置显示。视觉与控制的交互。
%数据链反向交互操作:地面站传输命令给飞控,飞控通过串口发给TX2,TX2调整检测的目标或程序相关设置。
%==============================

\backmatter
\bibliographystyle{nputhesis}
\bibliography{ref}

\Appendix
This is appendix.

\Thanks
This is a thanks.

\Work
% TODO 如何直接引用使参考文献的内容显示在这里

\statement
\end{document}

