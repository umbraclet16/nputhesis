% 中文摘要
\begin{abstract}  
旋翼无人机是一种垂直起降的无人飞行器,由于其机动灵活、结构简单、成本较低,在民用和军事领域有广泛的应用,并且逐渐向自主化、智能化的方向发展。为了适应各种复杂的任务,常常采用视觉技术来提高无人机的环境感知与交互能力。双目立体视觉技术通过得到周围视觉信息,经过图像采集、校正、立体匹配等步骤得到视差结果,进而重建出空间景物的三维信息,通过立体视觉技术实现目标的识别与定位,具备不易发现、信息量全,且能获取深度信息等优势。因此,本文以旋翼无人机的目标检测与定位为研究目标,构建了双目立体视觉模型,完成目标检测和立体匹配算法设计,搭建六旋翼实验平台进行算法验证,实验结果表明了本文设计算法的有效性。本文主要研究工作如下:

1、建立了摄像机模型和双目立体视觉模型。首先根据摄像机成像过程定义了相关坐标系,给出了坐标变换关系,并建立了摄像机线性模型,然后考虑畸变因素,建立摄像机非线性模型。分析了双目立体视觉原理中的三角测量、对极几何等关键理论,推导了 Bouguet 立体校正算法和张正友摄像机标定算法的步骤,最后完成了 ZED 双目相机的标定实验,仿真结果表明具有较好的误差效果。

2、为了提高目标检测效率,提出了一种基于 YOLO 的改进目标检测算法。首先介绍了卷积神经网络的相关知识,然后分析了 YOLO 目标检测算法的策略与过程,针对算法中代价函数误差分散的问题,对卷积神经网络模型的代价函数和网络结构进行了修改,提高了算法的分辨率。使用 TensorFlow 搭建网络模型,在 PASCAL VOC 数据集上进行了训练,并给出了多类物体的检测结果。在GPU上的测试结果表明算法具有较好的检测效果和实时性。

3、为了提高图像匹配精度,建立了基于卷积神经⽹络模型DispNetC立体匹配算法。介绍了端到端的立体匹配模型DispNet及加入相关层的 DispNetC算法。使用TensorFlow搭建了DispNetC网络,分别在FlyingThings3D和KITTI 2015数据集上进行了训练和参数调节,分析了训练集大小对网络匹配精度的影响,给出了两个数据集上的测试结果并与SGM、SPS-St等算法进行了对比分析。测试结果显示DispNetC具有较高的匹配精度和较快的运算速度,并且得到增大训练数据量有利于提高匹配精度的结论。

4、搭建了旋翼无人机目标检测定位系统。首先介绍了硬件系统的组成,然后提出了融合目标检测与立体匹配结果进行目标定位的方案,利用GrabCut算法提取目标轮廓并确定目标中心,结合立体匹配结果,应用三角测量原理完成摄像机坐标系下的目标定位,通过坐标变换实现机体系和导航系下的目标定位。最后给出了实验结果,并进行了误差分析,实验结果表明了本文算法的有效性。

    \begin{keywords}
        双目视觉,立体匹配,目标定位,目标检测,旋翼无人机,卷积神经网络
    \end{keywords}
\end{abstract}

% 英文摘要
\begin{Abstract}

    \begin{Keywords}
        Binocular Vision, Stereo Matching, Object localization, Object Detection, Rotorcraft UAV, Convolutional Neural Network
    \end{Keywords}
\end{Abstract}