% 中文摘要
\begin{abstract}  
旋翼无人机是一种垂直起降的无人飞行器,其具备空中悬停的能力,且机动性出色、结构简单,在民用和军事领域都具有广泛的应用前景,受到了学术界和工业界的普遍关注。
近年来,旋翼无人机的控制、导航等技术日趋成熟,系统向自主化、智能化的方向发展,应用场景也逐渐明晰。
为了适应各种复杂的任务,需要提高无人机的环境感知与交互能力。
%各种复杂的任务对。。提出了更高的要求。
因此,本文以旋翼无人机的目标检测与定位为研究课题,重点研究基于立体视觉的目标定位技术,并对实物系统进行了设计与实现。主要研究工作如下:

1、建立了摄像机模型和双目立体视觉模型。首先根据摄像机成像过程定义了相关坐标系,给出了坐标变换关系,并建立了摄像机线性模型;然后考虑畸变因素,建立摄像机非线性模型。分析了双目立体视觉原理中的三角测量、对极几何等关键理论,推导了Bouguet立体校正算法和张正友摄像机标定算法的数学过程,并使用MATLAB工具箱完成了ZED双目相机的标定实验。

2、提出了基于YOLO的目标检测算法。首先介绍了卷积神经网络的相关知识,然后分析了YOLO目标检测算法的策略与过程。针对算法存在的一些问题,对网络模型的代价函数和网络结构进行了修改。使用TensorFlow搭建了网络模型,在PASCAL VOC数据集上进行了训练,并给出了多类物体的检测结果。算法在GPU上的运行速度在30Hz以上,能够满足实时应用的要求。

3、建立了DispNetC立体匹配网络模型。梳理了传统立体匹配算法,然后介绍了端到端的立体匹配模型DispNet及加入相关层的DispNetC。使用TensorFlow搭建了DispNetC网络,分别在FlyingThings3D和KITTI 2015数据集上进行了训练和微调,给出了两个数据集上的测试结果并与SGM、SPS-St等算法进行了对比,分析了算法的优缺点。测试结果显示DispNetC具有较高的匹配精度,而运行时间仅为0.06秒,远快于大多数的立体匹配算法。最后分析了训练集大小对网络匹配精度的影响,比较结果说明增大训练数据量有利于提高匹配精度。

4、搭建了旋翼无人机目标检测定位系统。首先介绍了硬件系统的组成,然后提出了融合目标检测与立体匹配结果进行目标定位的方案,利用GrabCut算法提取目标轮廓并确定目标中心,结合立体匹配结果,应用三角测量原理完成摄像机坐标系下的目标定位,再通过坐标变换实现机体系和导航系下的目标定位。最后给出了实验结果,并进行了误差分析。


    \begin{keywords}
        立体匹配,目标定位,目标检测,旋翼无人机,卷积神经网络
    \end{keywords}
\end{abstract}

% 英文摘要
\begin{Abstract}

    \begin{Keywords}
        Stereo Matching, Object localization, Object Detection, Rotorcraft UAV, Convolutional Neural Network
    \end{Keywords}
\end{Abstract}