% 中文摘要
\begin{abstract}  
旋翼无人机是一种垂直起降的无人飞行器,由于其机动灵活、结构简单、成本较低,在民用和军事领域有广泛的应用,并且逐渐向自主化、智能化的方向发展。为了适应各种复杂的任务,常常采用视觉技术来提高无人机的环境感知与交互能力。双目立体视觉技术模拟人类视觉原理,经过图像采集、校正、立体匹配等步骤得到视差结果,进而重建出空间景物的三维信息,通过立体视觉技术实现目标的识别与定位,具备效率高、不易发现、信息量全等优势。因此,本文以旋翼无人机的目标检测与定位为研究目标,构建了双目立体视觉模型,完成目标检测和立体匹配算法设计,搭建六旋翼实验平台进行算法验证,实验结果表明了本文设计算法的有效性。本文主要研究工作如下:

1、建立了摄像机模型和双目立体视觉模型。首先根据摄像机成像过程定义了相关坐标系,给出了坐标变换关系,并建立了摄像机线性模型,然后考虑畸变因素,建立摄像机非线性模型。分析了双目立体视觉原理中的三角测量、对极几何等关键理论,推导了Bouguet立体校正算法和张正友摄像机标定算法的步骤,最后完成了ZED双目相机的标定实验,实验结果表明标定具有较高的精度。

2、为了提高目标检测能力,提出了一种基于YOLO的改进目标检测算法。首先介绍了卷积神经网络的相关知识,然后分析了YOLO目标检测算法的策略与过程,针对算法中代价函数误差分散的问题,对卷积神经网络模型的代价函数进行了修改,并通过调整网络结构提高了算法的分辨率。使用TensorFlow搭建网络模型,在PASCAL VOC数据集上进行了训练,并给出了多类物体的检测结果。在GPU上的测试结果表明算法具有较好的检测效果和实时性。

3、为了提高图像匹配精度和效率,建立了基于卷积神经⽹络模型DispNetC的立体匹配算法。介绍了端到端的立体匹配模型DispNet及加入相关层的变种结构DispNetC。使用TensorFlow搭建了DispNetC网络,分别在FlyingThings3D和KITTI 2015数据集上进行了训练和参数调节,分析了训练集大小对网络匹配精度的影响,给出了两个数据集上的测试结果并与SGM、SPS-St等算法进行了对比分析。测试结果显示DispNetC具有较高的匹配精度和较快的运算速度,并且得到增大训练数据量有利于提高匹配精度的结论。

4、搭建了旋翼无人机目标检测定位系统。首先介绍了硬件系统的组成,然后提出了融合目标检测与立体匹配结果进行目标定位的方案,利用GrabCut算法提取目标轮廓并确定目标中心,结合立体匹配结果,应用三角测量原理完成摄像机坐标系下的目标定位,通过坐标变换实现机体系和导航系下的目标定位。最后给出了实验结果,并进行了误差分析,实验结果表明了本文算法的有效性。

    \begin{keywords}
        双目视觉,立体匹配,目标定位,目标检测,旋翼无人机,卷积神经网络
    \end{keywords}
\end{abstract}

% 英文摘要
\begin{Abstract}
Rotorcraft UAV is a Virtual-Takeoff-and-Landing(VTOL) unmanned aerial vehicle. Due to its flexibility, simple structure and low cost, it has been widely used in civil and military fields, and it is advancing towards more autonomous and intelligent. To adapt to various complex tasks, vision techniques are commonly adopted to improve the environmental perception and interaction ability of UAVs. By simulating human visual principles, binocular stereo vision technique obtains disparity through image capture, stereo rectification and stereo matching, then reconstructs three-dimensional information of space scenery. Detecting and localizing objects with stereo vision techniques has the advantage of high efficiency, concealment and comprehensive information. Therefore, this paper focuses on object detection and localization for rotorcraft UAVs, builds up binocular stereo vision model, designs object detection and stereo matching algorithms, and constructs a hexarotor experimental platform to validate the algorithms. The main research work of this paper is as follows:

1. Camera models and binocular stereo vision model are established. According to camera imaging process, a few coordinate systems are defined and their transformations are presented. A linear model of camera is built up, after which distortions are considered and a nonlinear model is proposed. Critical theories in stereo vision such as triangulation and epipolar geometry are analyzed. Bouguet's stereo rectification algorithm and Zhang's camera calibration algorithm are derived. Finally, the calibration experiment on ZED binocular camera is conducted, and result shows good calibration accuracy.

2. To improve object detection capability, an modified algorithm based on YOLO is proposed. First, basic knowledge on convolutional neural network is introduced. Then the strategy and process of YOLO algorithm is analyzed. Due to the problem of error dispersion, the cost function of the model is modified. Besides, the network structure is adjusted to improve the resolution of detection. Afterwards, the model is set up with TensorFlow and trained on PASCAL VOC dataset. The test results on multiple classes of objects are given, demonstrating the good performance of the algorithm.
% 算法具有较好的检测效果和实时性。怎么翻译。。

3. In order to increase the matching accuracy and efficiency, a stereo matching algorithm based on CNN model DispNetC is presented. The end-to-end matching model DispNet and its variant DispNetC with correlation layer are introduced. Then DispNetC network is built using TensorFlow, trained and finetuned on FlyingThings3D and KITTI 2015 dataset respectively. Test results on both datasets are presented and compared with other algorithms such as SGM and SPS-St. Result shows that DispNetC has high matching accuracy and fast computing speed. It is also concluded that increasing the amount of training data is beneficial to improving the performance of the model.

4. A rotorcraft UAV object detection and localization system is established. The hardware composition is first introduced. Then the strategy to achieve object localization by fusing the result of object detection and stereo matching is proposed. Using GrabCut algorithm, the object contour is extracted and then its center is determined. Combined with the result of stereo matching, object coordinate in camera coordinate system is obtained with triangulation. Afterwards, object localization in body frame and navigation frame is achieved through coordinate transformation. Experimental results are given and error analysis is performed. Result proves the effectiveness of the proposed algorithm.


    \begin{Keywords}
        Binocular Vision, Stereo Matching, Object localization, Object Detection, Rotorcraft UAV, Convolutional Neural Network
    \end{Keywords}
\end{Abstract}






